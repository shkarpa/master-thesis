\chapter*{Conclusion}
\addcontentsline{toc}{chapter}{Conclusion}

Skew diagrams lack some of the important properties of moon polyominoes (e.g. comparability) which makes the fillings of skew diagrams
behave in more complicated ways and thus not as much progress has been made in the past years regarding fillings of skew diagrams.
The aim of the present thesis is to extend the current knowledge of skew diagrams and perhaps find new approaches to proving
facts about them.

This thesis presents two original results which were the product of the research conducted during author's master studies.
In the first half of the thesis, we deal with sparse fillings of skew diagrams and attempt to prove results similar
to what is known about sparse fillings of Ferrers diagrams.
Theorem \ref{thm_main} is a partial result and an initial step towards proving a more general hypothesis, which
could be used to prove new enumerative results about singleton classes, similar to the work of Backelin, West and Xin \cite{Backelin07}.

\begin{hypo}
Given any skew diagram $S$ and an integer $l \geq 1$, the number of sparse fillings of $S$ avoiding a $NE$-chain of length $l$
is greater or equal to the number of sparse fillings of $S$ avoiding a $SE$-chain of length $l$.
\end{hypo}

We have shown that for many skew diagrams, the inequality holds, even with equality. The author hopes that this work will be useful
in the future attempts to prove the general result.


In the second half of the thesis some progress was made considering general 0-1-fillings of skew diagrams instead of just sparse fillings.
The proof of Theorem \ref{thm_2} is a direct generalisation of Theorem \ref{thm_1} and introduces an interesting iterative proof method
which might also find its use elsewhere in the field.

