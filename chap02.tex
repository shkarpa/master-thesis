\chapter{A bijection between fillings of special skew diagrams}

\section{Main result}

Let $S^{forb}$ be the forbidden skew diagram $((3,3,2),(1))$.

\begin{figure}[h]
\centering
\begin{tikzpicture}[line cap=round,line join=round,>=triangle 45,x=0.75cm,y=0.75cm]
\clip(5.9,0.53) rectangle (9.1,4.25);
\draw (6,3)-- (6,1);
\draw (7,4)-- (7,1);
\draw (8,4)-- (8,1);
\draw (9,4)-- (9,2);
\draw (9,2)-- (6,2);
\draw (8,1)-- (6,1);
\draw (6,3)-- (9,3);
\draw (9,4)-- (7,4);
\end{tikzpicture}
\caption{$S^{forb}$}
\end{figure}

\begin{lemma} \label{lemma_forb_rec}
A skew diagram $S$ avoids $S^{forb}$ if and only if for every cell $c$ of $S$ at least one
of the following conditions is satisfied:
\begin{itemize}
\item
either the set
of cells of $S$ to the left and above $c$ (including the cells in the same row or column as $c$)
forms a rectangle, or 
\item the set of cells of $S$ to the right and below of $c$ (including the cells in the same row or column
as $c$) forms a rectangle.
\end{itemize}
\end{lemma}
\begin{proof}
If $S$ does contain $S^{forb}$, then the cell that represents the middle cell of an occurence of $S^{forb}$ violates both conditions.
On the other hand, if a cell $c$ of $S$ violates both conditions, we find an occurence of $S^{forb}$ in $S$ with this cell as its middle cell.
Let $A$ be the maximal rectangle contained in $S$ which has $c$ as its northeast corner and let $B$ be the maximal rectangle contained
in $S$ which has $c$ as its southwest corner. Then the 7 cells in the corners of these two rectangles form an occurence of $S^{forb}$.
\end{proof}

We can now formulate the main result of this chapter.

\begin{thm} \label{thm_main}
Let $S$ be a skew diagram avoiding $S^{forb}$, let $l \geq 1$ be an integer and let $\bm{r}$ and $\bm{c}$ be sequences of 0's
and 1's of lengths $w(S)$ and $h(S)$ respectively. Then there is a bijection between 
the sets $\F^{NE}(S, l, \bm{r}, \bm{c})$ and $\F^{SE}(S, l, \bm{r}, \bm{c})$.
\end{thm}

To prove this theorem, we will utilize other known bijection theorems on Ferrers diagrams and moon polyominoes
described in sections \ref{sec3} and \ref{sec4}, but first,
we will prove a useful equivalent characterization of skew diagrams avoiding $S^{forb}$.

\section{Simple skew diagrams}

Let $S$ be a skew diagram and let a vertical line leading between two adjacent columns of $S$ divide it into two skew diagrams $S_1$ and
$S_2$. Then we say that $S$ is a \emph{vertical concatenation} of $S_1$ and $S_2$ and we write $S = S_1 |^v S_2$. Similarly we define
\emph{horizontal concatenation} and write $T = T_1 |^h T_2$. Note that there are mutiple ways how to concatenate two skew diagrams.

We say that $S$ is a \emph{simple skew diagram} if it can be written as 

$$S = F_1 |^v G_1 |^h F_2 |^v \cdots |^h F_n |^v G_n,$$

where all $F_i$ are $NW$ Ferrers diagrams, all $G_i$ are $SE$ Ferrers diagrams and all $F_i$ and $G_i$ are nonempty with the possible
exception of $G_n$. We call this representation of $S$ the \emph{simple representation}.

\begin{figure}[h]
\centering
\begin{tikzpicture}[line cap=round,line join=round,>=triangle 45,x=0.75cm,y=0.75cm]
\clip(-4.25,-4.33) rectangle (6.19,4.2);
\draw (-4,-4)-- (-4,0);
\draw (-4,0)-- (0,0);
\draw (0,0)-- (0,-2);
\draw (0,-2)-- (-1,-2);
\draw (-1,-2)-- (-1,-3);
\draw (-1,-3)-- (-2,-3);
\draw (-2,-3)-- (-2,-4);
\draw (-2,-4)-- (-4,-4);
\draw (0,0)-- (0,1);
\draw (0,1)-- (2,1);
\draw (2,1)-- (2,2);
\draw (0,-1)-- (4,-1);
\draw (2,2)-- (4,2);
\draw (4,2)-- (4,-1);
\draw (-2.8,-1.35) node[anchor=north west] {$F_1$};
\draw (1.64,0.42) node[anchor=north west] {$G_1$};
\draw (2,2)-- (5,2);
\draw (3,2)-- (3,4);
\draw (3,4)-- (6,4);
\draw (6,4)-- (6,3);
\draw (6,3)-- (5,3);
\draw (5,3)-- (5,2);
\draw (3.5,3.35) node[anchor=north west] {$F_2$};
\end{tikzpicture}
\caption{A simple skew skew diagram}
\end{figure}

The key lemma follows.

\begin{lemma} \label{lemma_key}
A skew diagram is simple if and only if it does not contain $S^{forb}$.
\end{lemma}
\begin{proof}
In a simple skew diagram $S$ every cell is in either a $NW$ Ferrers subdiagram or a $SE$ Ferrers subdiagram and thus satisfies
at least one of the conditions
of Lemma \ref{lemma_forb_rec}. Therefore $S$ avoids $S^{forb}$. 

We now prove that every skew diagram avoiding $S^{forb}$ is simple by induction on the number of turns on the upper border of a skew diagram. 
If there is only one turn on the upper border of a skew diagram, then it is necessarily a $NW$ Ferrers diagram and there is nothing to prove.
Let $S$ be a skew diagram whose upper border has at least two turns.

Let $u_1$ and $r_1$ be the first two segments of the upper border of $S$, starting from the lower left corner, with $u_1$ going up
and $r_1$ going to the right. Together with the corresponding part of the lower border they bound a $NW$ Ferrers diagram
$F$ which can be separated by a vertical line $v$ from the rest of $S$, as we can see in Figure \ref{figure_key_lemma}.

Next, consider the cell $c$ attached to the upper right corner of $F$ as indicated in Figure \ref{figure_key_lemma}.
Since $S$ avoids $S^{forb}$ and the upper right corner of $c$ is a turn of the border of $S$, we get by Lemma \ref{lemma_forb_rec}
that the part of $S$ to the left and below of $c$ must form a rectangle, bounded from below by the lower border segment $r_2$
and from the right by the lower border segment $u_2$. These two segments of the lower border, together with
the corresponding part of the upper border of $S$ bound a $SE$ Ferrers diagram $G$, which can again be separated from the 
rest of $S$ by a horizontal line $h$.

Finally, let $S'$ be the part of $S$ above the line $h$. Then $S'$ is a skew diagram which definitely has fewer turns
on its upper border than $S$, so the induction applies and we get that $S'$ can be written as

$$S' = F_1 |^v G_1 |^h F_2 |^v \cdots |^h F_n |^v G_n,$$

where $F_1, \ldots, F_n$ are nonempty $NW$ Ferrers diagrams and $G_1, \ldots, G_n$ are $SE$ Ferrers diagrams. Together
with the constructed diagrams $F$ and $G$ we get that

$$S = F |^v G |^h F_1 |^v G_1 |^h F_2 |^v \cdots |^h F_n |^v G_n,$$

and thus $S$ is simple.

\begin{figure}[h]
\centering
\begin{tikzpicture}[line cap=round,line join=round,>=triangle 45,x=1.0cm,y=1.0cm]
\clip(1.57,2) rectangle (15.59,10.27);
\fill[fill=black,fill opacity=0.1] (7,7) -- (7,6.68) -- (7.32,6.68) -- (7.32,7) -- cycle;
\draw (3,3)-- (3,7);
\draw (3,7)-- (7,7);
\draw (7,10)-- (7,1);
\draw [shift={(3.34,6.88)},dash pattern=on 2pt off 2pt]  plot[domain=4.62:5.93,variable=\t]({1*3.89*cos(\t r)+0*3.89*sin(\t r)},{0*3.89*cos(\t r)+1*3.89*sin(\t r)});
\draw (7,6)-- (10,6);
\draw (10,6)-- (10,9);
\draw (4,9)-- (15,9);
\draw [shift={(9.6,6.48)},dash pattern=on 2pt off 2pt]  plot[domain=1.88:2.94,variable=\t]({1*2.65*cos(\t r)+0*2.65*sin(\t r)},{0*2.65*cos(\t r)+1*2.65*sin(\t r)});
\draw (7,7)-- (7,6.68);
\draw (7,6.68)-- (7.32,6.68);
\draw (7.32,6.68)-- (7.32,7);
\draw (7.32,7)-- (7,7);
\begin{scriptsize}
\draw[color=black] (2.75,5.11) node {\normalsize $u_1$};
\draw[color=black] (4.85,7.29) node {\normalsize $r_1$};
\draw[color=black] (7.14,4.38) node {\normalsize $v$};
\draw[color=black] (8.56,5.86) node {\normalsize $r_2$};
\draw[color=black] (10.19,7.62) node {\normalsize $u_2$};
\draw[color=black] (10.8,9.23) node {\normalsize $h$};
\draw[color=black] (7.74,6.72) node {\normalsize $c$};
\draw[color=black] (4.8,5.2) node {\large $F$};
\draw[color=black] (8.5,7.5) node {\large $G$};
\end{scriptsize}
\end{tikzpicture}
\caption{}
\label{figure_key_lemma}
\end{figure}
\end{proof}

\section{The bijection on Ferrers diagrams} \label{sec3}
Krattenthaler \cite{Krattenthaler06} proved Theorem \ref{thm_main} for the special case of Ferrers diagrams using the growth diagram
construction developed by Britz and Fomin \cite{Fomin01}. We briefly introduce growth diagrams
and reformulate the results of this approach in our setting. For convenience, we will describe the construction for $NW$
Ferrers diagrams only, however, the modification for $SE$ diagrams is obvious and we shall use it without stating it properly.

\begin{figure}[h]
\centering
\begin{tikzpicture}[line cap=round,line join=round,>=triangle 45,x=1.0cm,y=1.0cm]
\clip(0,0) rectangle (3,3);
\draw (1,1)--(1,2);
\draw (1,2)--(2,2);
\draw (2,2)--(2,1);
\draw (2,1)--(1,1);
\draw (0.8, 0.8) node {$\nu$};
\draw (0.8, 2.2) node {$\lambda$};
\draw (2.2, 2.2) node {$\mu$};
\draw (2.2, 0.8) node {$\rho$};
\end{tikzpicture}
\caption{A cell of a growth diagram}
\label{figure_growth_cell}
\end{figure}

Let $F$ be a Ferrers diagram filled with a sparse filling. 
We will inductively label every corner of every cell of $F$ by a partition, obtaining a \emph{growth diagram}.
We start by labeling the corners on the left and upper borders by the empty partition $\emptyset$. If a cell
already has all corners except the lower-right corner labeled by partitions $\lambda$, $\mu$, and $\nu$ as in Figure \ref{figure_growth_cell},
we construct the remaining partition $\rho$ by the following \emph{forward local} rules:
\begin{itemize}
\item [(F1)] If the cell does not contain a 1 and $\lambda = \mu = \nu$, then $\rho = \lambda$.
\item [(F2)] If the cell does not contain a 1 and $\mu \neq \nu$, then $\rho = \mu \cup \nu$.
\item [(F3)] If the cell does not contain a 1, $\lambda \subsetneq \mu = \nu$ and $\lambda$ and $\mu$ differ in their $i$-th part,
then we obtain $\rho$ by increasing the $i+1$-th part of $\mu$ by 1.
\item [(F4)] If the cell contains a 1, then necessarily $\lambda = \nu = \mu$ and we obtain $\rho$ by increasing the first
part of $\lambda$ by 1. 
\end{itemize}

Note that the equality of partitions in (F4) is implied by the fact that there is no 1 in the row of cells to the left or 
in the column of cells above the considered cell.

\begin{figure}
\centering
\begin{tikzpicture}[line cap=round,line join=round,>=triangle 45,x=1.0cm,y=1.0cm]
\clip(0,0) rectangle (6,6);

\draw (1,1)--(1,5);
\draw (2,1)--(2,5);
\draw (3,1)--(3,5);
\draw (4,2)--(4,5);
\draw (5,3)--(5,5);


\draw (1,1)--(3,1);
\draw (1,2)--(4,2);
\draw (1,3)--(5,3);
\draw (1,4)--(5,4);
\draw (1,5)--(5,5);

\draw (1.5, 3.5) node {\large $\times$};
\draw (2.5, 2.5) node {\large $\times$};
\draw (3.5, 4.5) node {\large $\times$};

\draw (1.2, 1.2) node {\footnotesize $\emptyset$};
\draw (1.2, 2.2) node {\footnotesize $\emptyset$};
\draw (1.2, 3.2) node {\footnotesize $\emptyset$};
\draw (1.2, 4.2) node {\footnotesize $\emptyset$};
\draw (1.2, 5.2) node {\footnotesize $\emptyset$};
\draw (2.2, 1.2) node {\footnotesize $1$};
\draw (2.2, 2.2) node {\footnotesize $1$};
\draw (2.2, 3.2) node {\footnotesize $1$};
\draw (2.2, 4.2) node {\footnotesize $\emptyset$};
\draw (2.2, 5.2) node {\footnotesize $\emptyset$};
\draw (3.2, 1.2) node {\footnotesize $2$};
\draw (3.2, 2.2) node {\footnotesize $2$};
\draw (3.2, 3.2) node {\footnotesize $1$};
\draw (3.2, 4.2) node {\footnotesize $\emptyset$};
\draw (3.2, 5.2) node {\footnotesize $\emptyset$};
\draw (4.2, 2.2) node {\footnotesize $21$};
\draw (4.2, 3.2) node {\footnotesize $11$};
\draw (4.2, 4.2) node {\footnotesize $1$};
\draw (4.2, 5.2) node {\footnotesize $\emptyset$};
\draw (5.2, 3.2) node {\footnotesize $11$};
\draw (5.2, 4.2) node {\footnotesize $1$};
\draw (5.2, 5.2) node {\footnotesize $\emptyset$};



\end{tikzpicture}
\caption{An example of a growth diagram}
\label{figure_growth_diagram}
\end{figure}


Let $w=w_1w_2\cdots w_k$ be an $R$-$U$ sequence. 
An \emph{oscillating tableau of type $w$ and shape $\emptyset/\emptyset$} 
is a sequence of partitions $(\lambda^0, \lambda^1, \ldots, \lambda^k)$ with the following properties:
\begin{itemize}
\item $\lambda^0 = \lambda^k = \emptyset$
\item For $1 \leq i \leq k$, the sizes of $\lambda^{i-1}$ and $\lambda^i$ differ by at most 1.
\item If $w_i = R$, then $\lambda^{i-1} \subseteq \lambda^i$.
\item If $w_i = U$, then $\lambda^{i-1} \supseteq \lambda^i$.
\end{itemize}

Remarkably, both the filling and the entire growth diagram of a Ferrers diagram of type $w$ can be reconstructed from
the sequence of partitions labeling the corners of its south-east border, which clearly is an oscillating tableau
of type $w$ and shape $\emptyset/\emptyset$. For more details we refer the reader to \cite[Section 2]{Krattenthaler06}.
The main consequence of this construction is the following theorem.

\begin{thm}[{\cite[Theorem 1]{Krattenthaler06}}] \label{thm_krat_1}
Let $F$ be a $NW$ Ferrers diagram of type $w = w_1w_2\ldots w_k$. The mapping, which to each sparse filling of $F$
assigns the oscillating tableau $(\emptyset = \lambda^0, \lambda^1, \ldots, \lambda^k = \emptyset)$
constructed as shown above, is a bijection between the set of all sparse fillings of $F$
and the set of all oscillating tableaux of type $w$ and shape $\emptyset/\emptyset$. In addition, this mapping has the following properties:
\begin{itemize}
\item $\lambda^{i-1} \subsetneq \lambda^i$ if and only if there is a $1$ in the column above the corners labeled by $\lambda^{i-1}$
and $\lambda^{i}$.
\item $\lambda^{i} \subsetneq \lambda^{i-1}$ if and only if there is a $1$ in the row to the left of the corners labeled by $\lambda^{i-1}$
and $\lambda^{i}$.
\end{itemize}
\end{thm}

For example, the filling of the Ferrers diagram in Figure \ref{figure_growth_diagram} is mapped to the oscillating tableau
$(\emptyset, 1, 2, 2, 21, 11, 11, 1, \emptyset)$. 

The very useful fact about growth diagrams 
is that the lengths of both longest $NE$-chains and $SE$-chains in a Ferrers
diagram can be deduced from partitions labeling the corners.

\begin{thm}[{\cite[Theorem 2]{Krattenthaler06}}] \label{thm_greene}
Let $F$ be a Ferrers diagram with a sparse filling and a growth diagram constructed accordingly and let $\lambda = (\lambda_1, \ldots, \lambda_k)$
be a partition labeling the southwest corner of a cell $c$. 
Let $R$ be the maximal rectangle contained in $F$ with the cell $c$ in its southwest corner.
Then the longest $NE$-chain contained in $R$ has length $k$ and the longest $SE$-chain contained in $R$ has length $\lambda_1$.
\end{thm}

Using growth diagrams and the theorems above it is now easy to prove Theorem \ref{thm_main} for Ferrers diagrams.

\begin{thm}[{\cite[Theorem 3]{Krattenthaler06}}] \label{thm_krat_3}
Let $F$ be a Ferrers diagram and $l \geq 1$ an integer. Let $\bm{r}$ and $\bm{c}$ be sequences of 0's and 1's
of length $w(F)$ and $h(F)$ respectively. 
Then there is a bijection between the sets $\F^{NE}(F, l, \bm{r}, \bm{c})$ and $\F^{SE}(F, l, \bm{r}, \bm{c})$.
\end{thm}
\begin{proof}
For given sequences $\bm{r}$ and $\bm{c}$ let $F'$ be the diagram obtained from $F$ by deleting the rows
with a 0 entry in $\bm{r}$ and columns with a 0 entry in $\bm{c}$. Since the deleted rows and columns
are always empty in the considered fillings, it is enough to construct a bijection
between the sets $\F^{NE}(F',l,\bm{1},\bm{1})$ and $\F^{SE}(F',l,\bm{1},\bm{1})$ where $\bm{1}$ is a sequence of only 1's
and this bijection is trivially extended to the sets in question.

Choose a filling of $F'$ from $\F^{NE}(F',l,\bm{1},\bm{1})$. The Theorem \ref{thm_krat_1} assigns to this filling
an oscillating tableau $(\emptyset = \lambda^0, \lambda^1, \ldots, \lambda^k = \emptyset)$ and by Theorem \ref{thm_greene}
every partition $\lambda^i$ has at most $l$ parts with equality occuring at least once. Now consider
the oscillating tableau $(\emptyset = (\lambda^0)^T, (\lambda^1)^T, \ldots, (\lambda^k)^T = \emptyset)$
obtained by transposing every partition of the original tableau. We use the Theorem \ref{thm_krat_1} again
to assign a sparse filling of $F'$ to the transposed tableau. Since the transposition does not change the size
of a partition, it is true that $(\lambda^{i-1})^T$ and $(\lambda^{i})^T$ differ by 1 in size for each $1 \leq i \leq k$
and therefore the obtained filling of $F'$ contains a 1 in each row and each column. In addition, each $(\lambda^i)^T$
satisfies $(\lambda^i)^T_1 \leq l$ with equality occuring at least once and so by Theorem \ref{thm_greene} the
length of the longest $SE$-chain in $F'$ is equal to $l$. Therefore, the obtained filling is in $\F^{SE}(F', l, \bm{1}, \bm{1})$.
\end{proof}

\section{A bijection on moon polyominoes} \label{sec4}

In his work, Rubey \cite{Rubey11} proves more general bijective results about fillings of moon polyominoes. In particular,
he shows that permuting the columns of a moon polyomino in any way such that it remains a moon polyomino does not change
the number of 0-1 fillings having a fixed length of the longest $NE$-chain and prescribed number of 1's in each row.
Here we formulate a part of his results which will be useful in our efforts. 

Given a finite sequence $s = (s_1, s_2, \ldots, s_n)$ and a permutation $\sigma$ of length $n$,
we denote by $\sigma s$ the sequence $(s_{\sigma(1)}, s_{\sigma(2)}, \ldots, s_{\sigma(n)})$. In addition,
given a moon polyomino $M$ and a permutation $\pi$ of length $w(M)$, we denote by $\sigma M$ the polyomino created by
permuting the columns of $M$ according to $\pi$. We will also need a stronger version of Definition \ref{def_rubey}.

\begin{defn}[{\cite[Definition 5.2]{Rubey11}}]\label{def_strong_rubey}
Let $M$ be a moon polyomino, $\bm{r}$, $\bm{c}$ sequences of 0's and 1's of lengths $h(M)$ and $w(M)$ respectively
and $\Lambda$ a mapping which assigns to every maximal rectangle $R$ in $M$ a positive integer $\Lambda(R)$.
Then $\F^{NE}(M, \Lambda, \bm{r}, \bm{c})$ is the set of all sparse fillings of $P$ such that
\begin{itemize}
\item there is a 1 in the $i$-th row of $P$ if and only if $r_i = 1$,
\item there is a 1 in the $i$-th column of $P$ if and only if $c_i = 1$,
\item for every maximal rectangle $R$ the length of the longest $NE$-chain in the filling of $R$ is equal to $\Lambda(R)$.
\end{itemize}
\end{defn}

Note that in a moon polyomino, every maximal rectangle is uniquely determined by its height and width. 
Indeed, consider two maximal rectangles of the same height in a moon polyomino $M$. Since the columns of the two rectangles
are comparable, they must in fact span the same rows, and since $M$ is convex, they must be contained in each other and thus
be identical.

\begin{thm}[{\cite[Theorem 5.3]{Rubey11}}]\label{thm_rubey}
Let $M$ be a moon polyomino and $R$ be a maximal rectangle in $M$ such that the column of $M$ containing 
the leftmost column $C$ of $R$ has the same height as $C$. Let $\sigma$ be the permutation of columns of $M$
which moves the column $C$ to the right end of $R$ and shifts the other columns intersecting $R$ one spot to the left.
Then the sets of maximal rectangles of $M$ and $\sigma M$ coincide and 
for any $\Lambda$, $\bm{r}$ and $\bm{c}$ there is a bijective map which maps every filling in $\F^{NE}(M, \Lambda, \bm{r}, \bm{c})$ to a 
filling in $\F^{NE}(\sigma M, \Lambda, \bm{r}, \bm{\sigma c})$.
\end{thm}

\begin{figure}
\centering
\subfloat[][$M$]{
\begin{tikzpicture}[line cap=round,line join=round,>=triangle 45,x=0.75cm,y=0.75cm]
\clip(0.86,1.79) rectangle (9.18,9.16);
\draw (3,8)-- (3,4);
\draw (3,4)-- (8,4);
\draw (8,4)-- (8,8);
\draw (8,8)-- (3,8);
\draw (4,8)-- (4,4);
\draw (5,8)-- (5,4);
\draw (6,8)-- (6,4);
\draw (7,8)-- (7,4);
\draw (4,7)-- (8,7);
\draw (4,6)-- (8,6);
\draw (4,5)-- (8,5);
\draw (3,7)-- (1,7);
\draw (1,7)-- (1,6);
\draw (1,6)-- (3,6);
\draw (2,7)-- (2,4);
\draw (2,4)-- (3,4);
\draw (2,5)-- (3,5);
\draw (4,4)-- (4,3);
\draw (4,3)-- (8,3);
\draw (8,3)-- (8,4);
\draw (7,3)-- (7,2);
\draw (7,2)-- (4,2);
\draw (4,2)-- (4,3);
\draw (5,8)-- (5,9);
\draw (5,9)-- (7,9);
\draw (7,9)-- (7,8);
\draw (6,8)-- (6,9);
\draw (8,7)-- (9,7);
\draw (9,7)-- (9,4);
\draw (9,4)-- (8,4);
\draw (5,4)-- (5,2);
\draw (6,4)-- (6,2);
\draw (8,6)-- (9,6);
\draw (8,5)-- (9,5);
\draw (7,4)-- (7,3);
\draw (3.5, 6) node {$C$};
\draw [line width=1.6pt] (3,4) -- (3,8);
\draw [line width=1.6pt] (8,8) -- (3,8);
\draw [line width=1.6pt] (8,8) -- (8,4);
\draw [line width=1.6pt] (3,4) -- (8,4);
\end{tikzpicture}
}
\subfloat[][$\sigma M$]{
\begin{tikzpicture}[line cap=round,line join=round,>=triangle 45,x=0.75cm,y=0.75cm]
\clip(2.74,0.77) rectangle (11.15,8.17);
\draw (5,7)-- (5,3);
\draw (5,3)-- (10,3);
\draw (10,3)-- (10,7);
\draw (10,7)-- (5,7);
\draw (6,7)-- (6,3);
\draw (7,7)-- (7,3);
\draw (8,7)-- (8,3);
\draw (9,7)-- (9,3);
\draw (5,6)-- (3,6);
\draw (3,6)-- (3,5);
\draw (3,5)-- (5,5);
\draw (4,6)-- (4,3);
\draw (4,3)-- (5,3);
\draw (4,4)-- (5,4);
\draw (10,6)-- (11,6);
\draw (11,6)-- (11,3);
\draw (11,3)-- (10,3);
\draw (10,5)-- (11,5);
\draw (10,4)-- (11,4);
\draw (5,6)-- (9,6);
\draw (5,5)-- (9,5);
\draw (5,4)-- (9,4);
\draw (6,7)-- (6,8);
\draw (6,8)-- (8,8);
\draw (8,8)-- (8,7);
\draw (7,8)-- (7,7);
\draw (5,3)-- (5,1);
\draw (5,1)-- (8,1);
\draw (8,1)-- (8,3);
\draw (5,2)-- (9,2);
\draw (9,2)-- (9,3);
\draw (7,1)-- (7,3);
\draw (6,3)-- (6,1);
\draw (9.5, 5) node {$C$};
\draw [line width=1.6pt] (5,3) -- (10,3);
\draw [line width=1.6pt] (10,3) -- (10,7);
\draw [line width=1.6pt] (10,7) -- (5,7);
\draw [line width=1.6pt] (5,7) -- (5,3);
\end{tikzpicture}
}
\caption{The operation described by Theorem \ref{thm_rubey}}
\end{figure}
Of course, since the mapping described by the theorem is bijective, the inverse mapping is also a bijection and therefore
we may transform moon polyominoes by moving the rightmost column of a maximal rectangle instead of the leftmost.
Furthermore, due to symmetry we can also use this result for moving rows instead of columns.

We follow up by using the theorem above to prove a bijection between sparse fillings of Ferrers diagrams which satisfy
two different sets of additional constraints along with having the length of the longest $NE$-chain equal to $l \geq 1$.
Consider a $NW$ Ferrers diagram $F$ with at least $n$ longest columns and at least $m$ longest rows and let $A$
be the rectangle consisting of the $n$ leftmost columns of $F$ and let $B$ be the rectangle consisting of top $m$ rows of $F$.
Let $\bm{a} = (a_1, a_2, \ldots, a_n)$ and $\bm{b} = (b_1, b_2, \ldots, b_m)$ be two weakly increasing sequences of nonnegative integers satisfying
$a_n \leq l$ and $b_m \leq l$. Given any sequences $\bm{r}$ and $\bm{c}$ as in Definition \ref{def_rubey}, we define the following
sets of sparse fillings of $F$:
\begin{itemize}
\item $\F^{NE}_{in}(F, l, \bm{r}, \bm{c}, \bm{a}, \bm{b})$ is the set of all fillings from
$\F^{NE}(M, l, \bm{r}, \bm{c})$ satisfying for every $1 \leq i \leq n$ that the length of the longest $NE$-chain inside the
rectangle consisting of $i$ \emph{rightmost} columns of $A$ is equal to $a_i$ and for every $1 \leq j \leq m$ and the length of the longest $NE$-chain 
inside the rectangle consisting of $j$ \emph{bottom} rows of $B$ is equal to $b_j$,

\item $\F^{NE}_{out}(F, l, \bm{r}, \bm{c}, \bm{a}, \bm{b})$ is the set of all fillings from
$\F^{NE}(M, l, \bm{r}, \bm{c})$ satisfying for every $1 \leq i \leq n$ that the length of the longest $NE$-chain inside
the rectangle consisting of $i$ \emph{leftmost} columns of $A$ is equal to $a_i$ and for every $1 \leq j \leq m$ the length of the longest $NE$-chain 
inside the rectangle consisting of $j$ \emph{top} rows of $B$ is equal to $b_j$.
\end{itemize}

Similarly we define the sets $\F^{SE}_{in}(F,l, \bm{r}, \bm{c}, \bm{a}, \bm{b})$ and $\F^{SE}_{out}(F,l, \bm{r}, \bm{c}, \bm{a}, \bm{b})$
with the constraints imposed the same way as above but on lengths of $SE$-chains instead of $NE$-chains.
We now show that the two described sets of fillings are actually of the same size.

\begin{lemma} \label{lemma_constraints}
Let $F$ be a $NW$ Ferrers diagram with the sequences $\bm{a}$ and $\bm{b}$ of constraints on fillings as described above. Then
there is a bijection between the sets $\F^{NE}_{in}(F, l, \bm{r}, \bm{c}, \bm{a}, \bm{b})$ and $\F^{NE}_{out}(F, l, \bm{r}, \bm{c}, \bm{a}, \bm{b})$.
\end{lemma}

\begin{proof}

We start by modifying the diagram $F$ by adding some new rows and columns, adjusting $\bm{r}$ and $\bm{c}$ appropriately by assigning
0's to the new rows and columns, which will therefore always remain empty. The modification is as follows:
\begin{itemize}
\item for $i$ iterating from $n$ to $1$, attach a new row of length $i$ below the $i$ rightmost columns of $A$,
\item for $j$ iterating from $m$ to $1$, attach a new column of length $j$ to the right of the bottom $j$ rows of $B$.
\end{itemize}
As a result of this modification we obtain a moon polyomino $M_1$ (see Figure \ref{figure_M1}) and sequences $\bm{r^1}$ and $\bm{c^1}$.

\begin{figure}[h]
\centering
\begin{tikzpicture}[line cap=round,line join=round,>=triangle 45,x=0.9cm,y=0.9cm]
\clip(-4.17,-6.58) rectangle (10.6,6.3);
\draw [line width=1.6pt] (-4,6)-- (-4,-3);
\draw (-4,-3)-- (0,-3);
\draw [line width=1.6pt] (-4,6)-- (7,6);
\draw (7,6)-- (7,2);
\draw (7,2)-- (5,2);
\draw (5,2)-- (5,1);
\draw (5,1)-- (4,1);
\draw (4,1)-- (4,-1);
\draw (4,-1)-- (1,-1);
\draw (1,-1)-- (1,-2);
\draw (1,-2)-- (0,-2);
\draw (0,-2)-- (0,-3);
\draw [line width=1.6pt] (-0.5,-3)-- (-0.5,6);
\draw [line width=1.6pt] (7,2.5)-- (-4,2.5);
\draw [line width=1.6pt] (-4,-3)-- (-0.5,-3);
\draw [line width=1.6pt] (7,2.5)-- (7,6);
\draw (-2.59,4.64) node[anchor=north west] {$F$};
\draw (2.22,4.64) node[anchor=north west] {$B$};
\draw (-2.55,-0.07) node[anchor=north west] {$A$};
\draw [dash pattern=on 2pt off 2pt] (-4,-3)-- (-4,-3.5);
\draw [dash pattern=on 2pt off 2pt] (-4,-3.5)-- (-3.5,-3.5);
\draw [dash pattern=on 2pt off 2pt] (-3.5,-3.5)-- (-3.5,-4);
\draw [dash pattern=on 2pt off 2pt] (-3.5,-4)-- (-3,-4);
\draw [dash pattern=on 2pt off 2pt] (-3,-4)-- (-3,-4.5);
\draw [dash pattern=on 2pt off 2pt] (-3,-4.5)-- (-2.5,-4.5);
\draw [dash pattern=on 2pt off 2pt] (-2.5,-4.5)-- (-2.5,-5);
\draw [dash pattern=on 2pt off 2pt] (-2.5,-5)-- (-2,-5);
\draw [dash pattern=on 2pt off 2pt] (-2,-5)-- (-2,-5.5);
\draw [dash pattern=on 2pt off 2pt] (-2,-5.5)-- (-1.5,-5.5);
\draw [dash pattern=on 2pt off 2pt] (-1.5,-5.5)-- (-1.5,-6);
\draw [dash pattern=on 2pt off 2pt] (-1.5,-6)-- (-1,-6);
\draw [dash pattern=on 2pt off 2pt] (-1,-6)-- (-1,-6.5);
\draw [dash pattern=on 2pt off 2pt] (-1,-6.5)-- (-0.5,-6.5);
\draw [dash pattern=on 2pt off 2pt] (-0.5,-6.5)-- (-0.5,-3);
\draw [dash pattern=on 2pt off 2pt] (7,6)-- (7.5,6);
\draw [dash pattern=on 2pt off 2pt] (7.5,6)-- (7.5,5.5);
\draw [dash pattern=on 2pt off 2pt] (7.5,5.5)-- (8,5.5);
\draw [dash pattern=on 2pt off 2pt] (8,5.5)-- (8,5);
\draw [dash pattern=on 2pt off 2pt] (8,5)-- (8.5,5);
\draw [dash pattern=on 2pt off 2pt] (8.5,5)-- (8.5,4.5);
\draw [dash pattern=on 2pt off 2pt] (8.5,4.5)-- (9,4.5);
\draw [dash pattern=on 2pt off 2pt] (9,4.5)-- (9,4);
\draw [dash pattern=on 2pt off 2pt] (9,4)-- (9.5,4);
\draw [dash pattern=on 2pt off 2pt] (9.5,4)-- (9.5,3.5);
\draw [dash pattern=on 2pt off 2pt] (9.5,3.5)-- (10,3.5);
\draw [dash pattern=on 2pt off 2pt] (10,3.5)-- (10,3);
\draw [dash pattern=on 2pt off 2pt] (10,3)-- (10.5,3);
\draw [dash pattern=on 2pt off 2pt] (10.5,3)-- (10.5,2.5);
\draw [dash pattern=on 2pt off 2pt] (10.5,2.5)-- (7,2.5);
\end{tikzpicture}
\caption{The moon polyomino $M_1$ after modifying $F$}
\label{figure_M1}
\end{figure}

We continue by defining a family $\mathcal{L}$ of mappings assigning a positive integer to every maximal rectangle of $M_1$.
A mapping $\Lambda$ belongs to $\mathcal{L}$ if and only if the following conditions are met:
\begin{enumerate}[(a)]
\item For $1 \leq i \leq n$, let $A_i$ be the unique maximal rectangle of width $i$ in $M_1$. Then $\Lambda(A_i) = a_i$.
\item For $1 \leq j \leq m$, let $B_j$ be the unique maximal rectangle of height $j$ in $M_1$. Then $\Lambda(B_j) = b_j$,
\item for any other maximal rectangle $R$ it is true that $\Lambda(R) \leq l$ 
\item There is at least one maximal rectangle $R$ such that $\Lambda(R) = l$.
\end{enumerate}

Let $\mathcal{F}'$ be the set of fillings of $M_1$ obtained as a union of $\mathcal{F}(M_1, \Lambda, \bm{r}^1, \bm{c}^1)$ for every
$\Lambda \in \mathcal{L}$. Then $\mathcal{F}'$ is the set of fillings of $M_1$ in which the longest $NE$-chain has length $l$
and in addition for every $1 \leq i \leq n$ the longest $NE$-chain in the rectangle $A_i$ has length $a_i$
and for every $1 \leq j \leq m$ the longest $NE$-chain in the rectangle $B_j$ has length $b_j$. But since the newly attached
rows and columns are always empty, the chains are always contained in the original diagram $F$ and therefore the set
$\mathcal{F}'$ is in 1-to-1 correspondence with the set $\mathcal{F}^NE_{in}(F, l, \bm{r}, \bm{c}, \bm{a}, \bm{b}$.

Next we perform the transformation described in Theorem \ref{thm_rubey} several times to obtain a Ferrers diagram $F_1$:
\begin{itemize}
\item for every $i$ iterating from $n$ to $1$, move the leftmost column of $A_i$ to the rightmost end of the rectangle,
\item for every $j$ iterating from $m$ to $1$, move the top row of $B_j$ to the bottom of the rectangle.
\end{itemize}
This creates the Ferrers diagram $F_1$ as illustrated by the Figure \ref{figure_F1}.
Let $\sigma$ be the permutation of columns and $\pi$ be the permutation of rows which created $F_1$ from $M_1$. By
Theorem \ref{thm_rubey} we get that for every $\Lambda \in \mathcal{L}$ there is a bijection between 
$\mathcal{F}^{NE}(M_1, \Lambda, \bm{r}^1, \bm{c}^1)$ and $\mathcal{F}^{NE}(F_1, \Lambda, \pi\bm{r}^1, \sigma\bm{c}^1)$.
Let $\mathcal{F}''$ be the set of fillings of $F_1$ obtained as a union of $\mathcal{F}^{NE}(F_1, \Lambda, \pi\bm{r}^1, \sigma\bm{c}^1)$
for every $\Lambda \in \mathcal{L}$. Overall we obtain a bijection between the sets $\mathcal{F}'$ and $\mathcal{F}''$.
Finally, notice that, similarly as the fillings of $\mathcal{F}'$ correspond to the fillings
of $\mathcal{F}^{NE}_{in}(F, l, \bm{r}, \bm{c}, \bm{a}, \bm{b})$,
also the fillings of $\mathcal{F}''$ correspond to the fillings of $\mathcal{F}^{NE}_{out}(F, l, \bm{r}, \bm{c}, \bm{a}, \bm{b})$,
which completes the proof.

\begin{figure}[h]
\centering
\begin{tikzpicture}[line cap=round,line join=round,>=triangle 45,x=0.9cm,y=0.9cm]
\clip(-4.17,-6.63) rectangle (10.58,6.13);
\draw [line width=1.6pt] (-4,6)-- (-4,-3);
\draw (-4,-3)-- (0,-3);
\draw [line width=1.6pt] (-4,6)-- (7,6);
\draw (7,6)-- (7,2);
\draw (7,2)-- (5,2);
\draw (5,2)-- (5,1);
\draw (5,1)-- (4,1);
\draw (4,1)-- (4,-1);
\draw (4,-1)-- (1,-1);
\draw (1,-1)-- (1,-2);
\draw (1,-2)-- (0,-2);
\draw (0,-2)-- (0,-3);
\draw [line width=1.6pt] (-0.5,-3)-- (-0.5,6);
\draw [line width=1.6pt] (7,2.5)-- (-4,2.5);
\draw [line width=1.6pt] (-4,-3)-- (-0.5,-3);
\draw [line width=1.6pt] (7,2.5)-- (7,6);
\draw (-2.58,4.62) node[anchor=north west] {$F$};
\draw (2.23,4.62) node[anchor=north west] {$B$};
\draw (-2.56,-0.08) node[anchor=north west] {$A$};
\draw [dash pattern=on 2pt off 2pt] (-4,-3)-- (-4,-6.5);
\draw [dash pattern=on 2pt off 2pt] (-4,-6.5)-- (-3.5,-6.5);
\draw [dash pattern=on 2pt off 2pt] (-3.5,-6.5)-- (-3.5,-6);
\draw [dash pattern=on 2pt off 2pt] (-3.5,-6)-- (-3,-6);
\draw [dash pattern=on 2pt off 2pt] (-3,-6)-- (-3,-5.5);
\draw [dash pattern=on 2pt off 2pt] (-3,-5.5)-- (-2.5,-5.5);
\draw [dash pattern=on 2pt off 2pt] (-2.5,-5.5)-- (-2.5,-5);
\draw [dash pattern=on 2pt off 2pt] (-2.5,-5)-- (-2,-5);
\draw [dash pattern=on 2pt off 2pt] (-2,-5)-- (-2,-4.5);
\draw [dash pattern=on 2pt off 2pt] (-2,-4.5)-- (-1.5,-4.5);
\draw [dash pattern=on 2pt off 2pt] (-1.5,-4.5)-- (-1.5,-4);
\draw [dash pattern=on 2pt off 2pt] (-1.5,-4)-- (-1,-4);
\draw [dash pattern=on 2pt off 2pt] (-1,-4)-- (-1,-3.5);
\draw [dash pattern=on 2pt off 2pt] (-1,-3.5)-- (-0.5,-3.5);
\draw [dash pattern=on 2pt off 2pt] (-0.5,-3.5)-- (-0.5,-3);
\draw [dash pattern=on 2pt off 2pt] (7,6)-- (10.5,6);
\draw [dash pattern=on 2pt off 2pt] (10.5,6)-- (10.5,5.5);
\draw [dash pattern=on 2pt off 2pt] (10.5,5.5)-- (10,5.5);
\draw [dash pattern=on 2pt off 2pt] (10,5.5)-- (10,5);
\draw [dash pattern=on 2pt off 2pt] (10,5)-- (9.5,5);
\draw [dash pattern=on 2pt off 2pt] (9.5,5)-- (9.5,4.5);
\draw [dash pattern=on 2pt off 2pt] (9.5,4.5)-- (9,4.5);
\draw [dash pattern=on 2pt off 2pt] (9,4.5)-- (9,4);
\draw [dash pattern=on 2pt off 2pt] (9,4)-- (8.5,4);
\draw [dash pattern=on 2pt off 2pt] (8.5,4)-- (8.5,3.5);
\draw [dash pattern=on 2pt off 2pt] (8.5,3.5)-- (8,3.5);
\draw [dash pattern=on 2pt off 2pt] (8,3.5)-- (8,3);
\draw [dash pattern=on 2pt off 2pt] (8,3)-- (7.5,3);
\draw [dash pattern=on 2pt off 2pt] (7.5,3)-- (7.5,2.5);
\draw [dash pattern=on 2pt off 2pt] (7.5,2.5)-- (7,2.5);
\end{tikzpicture}
\caption{The Ferrers diagram $F_1$ after applying Theorem \ref{thm_rubey}}
\label{figure_F1}
\end{figure}


%We start by a modification of the diagram $F$ which will maintain the fillings in question. Namely, we will add some new columns
%to the rectangle $A$ and modify $\bm{c}$ appropriately, assigning $0$ to the new columns. Then we will add some new
%rows to the rectangle $B$ and modify $\bm{r}$ appropriately. We modify $F$ as follows:
%\begin{itemize}
%\item add $l - a_n$ new columns to the left of $A$,
%\item for every $1 \leq i < n$, add $a_{i+1} - a_i$ new columns between the $(i+1)$-th and $i$-th columns of $A$, counting from the right side,
%\item add $l - b_m$ new rows above the modified rectangle $B$,
%\item for every $1 \leq j < m$, add $b_{j+1} - b_j$ new rows between the $(j+1)$-th and $j$-th rows of $B$, counting from below.
%\end{itemize}
%
%Let $s$ be the total number of added columns and let $t$ be the total number of added rows. Note that $A$ now has
%$n + s$ columns, $B$ has $m + t$ rows and the resulting diagram is still a Ferrers diagram $F'$. We further modify $F'$ by adding
%a rectangle $Q_A$ with $s$ new rows of length $n + s$ to the bottom of $F'$ and a rectangle $Q_B$
%with $t$ new columns of height $m + t$ to the right side of $F'$ so that the resulting diagram is a Ferrers diagram $F_1$.
%Finally, we add one new dummy column of height $h(F) + t$ left of the modified rectangle $A$ and a new dummy row of length $w(F) + s + 1$
%above the modified rectangle $B$, these will always remain empty. This addition results in a moon polyomino $M_1$.
%
%\begin{figure}[h]
%\centering
%\begin{tikzpicture}[line cap=round,line join=round,>=triangle 45,x=0.97cm,y=0.97cm]
%\clip(-5.51,-5.24) rectangle (9.11,7.6);
%\draw [line width=1.6pt] (-4,6)-- (-4,-3);
%\draw (-4,-3)-- (0,-3);
%\draw [line width=1.6pt] (-4,6)-- (7,6);
%\draw (7,6)-- (7,2);
%\draw (7,2)-- (5,2);
%\draw (5,2)-- (5,1);
%\draw (5,1)-- (4,1);
%\draw (4,1)-- (4,-1);
%\draw (4,-1)-- (1,-1);
%\draw (1,-1)-- (1,-2);
%\draw (1,-2)-- (0,-2);
%\draw (0,-2)-- (0,-3);
%\draw [line width=1.6pt] (-0.5,-3)-- (-0.5,6);
%\draw [line width=1.6pt] (7,2.5)-- (-4,2.5);
%\draw [dash pattern=on 2pt off 2pt] (7,2.5)-- (9,2.5);
%\draw [dash pattern=on 2pt off 2pt] (9,2.5)-- (9,6);
%\draw [dash pattern=on 2pt off 2pt] (9,6)-- (7,6);
%\draw [dash pattern=on 2pt off 2pt] (-4,-3)-- (-4,-5);
%\draw [dash pattern=on 2pt off 2pt] (-4,-5)-- (-0.5,-5);
%\draw [dash pattern=on 2pt off 2pt] (-0.5,-5)-- (-0.5,-3);
%\draw [line width=1.6pt] (-4,-3)-- (-0.5,-3);
%\draw [line width=1.6pt] (7,2.5)-- (7,6);
%\draw (1,1) node {$F$};
%\draw (2.23,4.64) node {$B$};
%\draw (-2.57,-0.06) node {$A$};
%\draw (8.07,4.45) node {$Q_B$};
%\draw (-2.44,-4.07) node {$Q_A$};
%\draw [dash pattern=on 2pt off 2pt] (-4,-5)-- (-5,-5);
%\draw [dash pattern=on 2pt off 2pt] (-5,-5)-- (-5,6);
%\draw [dash pattern=on 2pt off 2pt] (-5,6)-- (-4,6);
%\draw [dash pattern=on 2pt off 2pt] (-5,6)-- (-5,7);
%\draw [dash pattern=on 2pt off 2pt] (-5,7)-- (9,7);
%\draw [dash pattern=on 2pt off 2pt] (9,7)-- (9,6);
%\draw (-4.75,-4.75) node {$\times$};
%\draw (-4.25,-4.25) node {$\times$};
%\draw (-3.25,-3.75) node {$\times$};
%\draw (-1.75,-3.25) node {$\times$};
%\draw [dash pattern=on 2pt off 2pt] (-4,-3)-- (-5,-3);
%\draw (8.75,6.75) node {$\times$};
%\draw (8.25,6.25) node {$\times$};
%\draw (7.75,5.25) node {$\times$};
%\draw (7.25,3.75) node {$\times$};
%\draw [dash pattern=on 2pt off 2pt] (7,6)-- (7,7);
%\draw [dash pattern=on 2pt off 2pt] (-3.5,-3)-- (-3.5,6);
%\draw [dash pattern=on 2pt off 2pt] (-3,-3)-- (-3,6);
%\draw [dash pattern=on 2pt off 2pt] (-2,-3)-- (-2,6);
%\draw [dash pattern=on 2pt off 2pt] (-1.5,-3)-- (-1.5,6);
%\draw [dash pattern=on 2pt off 2pt] (7,5.5)-- (-5,5.5);
%\draw [dash pattern=on 2pt off 2pt] (7,5)-- (-5,5);
%\draw [dash pattern=on 2pt off 2pt] (7,4)-- (-5,4);
%\draw [dash pattern=on 2pt off 2pt] (7,3.5)-- (-5,3.5);
%
%\draw [dash pattern=on 2pt off 2pt] (-5.5,-3)-- (-5,-3);
%\draw [dash pattern=on 2pt off 2pt] (-5.5,-3)-- (-5.5,7);
%\draw [dash pattern=on 2pt off 2pt] (-5.5,7)-- (-5,7);
%\draw [dash pattern=on 2pt off 2pt] (-5.5,7)-- (-5.5,7.5);
%\draw [dash pattern=on 2pt off 2pt] (-5.5,7.5)-- (7,7.5);
%\draw [dash pattern=on 2pt off 2pt] (7,7)-- (7,7.5);
%\end{tikzpicture}
%\caption{The modified moon polyomino $M_1$}
%\label{figure_F1}
%\end{figure}
%
%We now set the entries of all the new rows and columns of $M_1$ in $\bm{r}$ and $\bm{c}$ to 1, except
%for the dummy row and column, getting
%the sequences $\bm{r^1}$ and $\bm{c^1}$, and fix the following partial sparse filling of $F_1$: 
%\begin{itemize}
%\item fill the $s \times s$ square consisting of the $s$ columns newly added to $A$ intersecting the $s$ rows of $Q_A$ with
%a $NE$-chain of length $s$,
%\item fill the $t \times t$ square consisting of the $t$ columns newly added to $B$ intersecting the $t$ rows of $Q_B$ with a
%$NE$-chain of length $t$.
%\end{itemize}
%
%The main observation now is that the fillings in $\F^{NE}(M_1, l, \bm{r^1}, \bm{c^1})$ with the described fixed filling
%of $Q_A$ and $Q_B$ are in 1-to-1 correspondence with the fillings of $\F^{NE}_{in}(F, l, \bm{r}, \bm{c}, \bm{a}, \bm{b})$.
%Indeed, consider a filling from $\F^{NE}_{in}(F, l, \bm{r}, \bm{c}, \bm{a}, \bm{b})$ and transform $F$ to $F_1$ whilst
%keeping the chosen filling in place. Since we added $1$'s only to newly added columns and rows, the resulting filling is still sparse
%and the constraints $\bm{a}$ and $\bm{b}$ accomodate for the $1$'s added in $Q_A$ and $Q_B$ and 
% ensure that the length of the longest $NE$-chain in $F_1$ is still $l$. Similarly if we take a filling of $\F^{NE}(M_1, l, \bm{r^1}, \bm{c^1})$
% with the described fixed filling of $Q_A$ and $Q_B$, the fixed part of the filling limits the lengths of longest $NE$-chains
% in $A$ and $B$ in the same way as the constraints $\bm{a}$ and $\bm{b}$, so by simply ignoring the added rows and columns we obtain
% a filling of $F$ from  $\F^{NE}_{in}(F, l, \bm{r}, \bm{c}, \bm{a}, \bm{b})$.
%
%We now apply the transformation described by Theorem \ref{thm_rubey} to $M_1$ multiple times. For every row of $F$ below the rectangle $B$,
%starting from the shortest, consider the corresponding row in $M_1$ and the maximal rectangle in $M_1$ containing it, and move the row to the 
%top of the rectangle. Then, for every column of $F$ right of $A$, starting from the shortest, consider the corresponding column in $F_1$
%and the maximal rectangle in $M_1$ containing it, and move the column to the left end of the rectangle. As a result we obtain
%a moon polyomino $M$ with rows permuted by $\pi$ and columns permuted by $\sigma$ when compared to $M_1$ and by Theorem \ref{thm_rubey}
%we obtain a bijection between $\F^{NE}(M_1, l, \bm{r^1}, \bm{c^1})$ and $\F^{NE}(M, l, \pi\bm{r^1}, \sigma\bm{c^1})$.
%
%Note that no cells of $Q_A$ or $Q_B$ were ever a part of any maximal rectangle used in the transformation 
%because of the extra dummy row and column we added. Therefore,
%since the filling changes only inside the maximal rectangle when using the map given by Theorem \ref{thm_rubey}, we may fix the
%filling of $Q_A$ and $Q_B$ as described above and we obtain a finer bijection between fillings of $\F^{NE}(F_1, l, \bm{r^1}, \bm{c^1})$
%with the fixed filling of $Q_A$ and $Q_B$ and fillings of $\F^{NE}(M, l, \pi\bm{r^1}, \sigma\bm{c^1})$ with the fixed filling of $Q_A$
%and $Q_B$.
%
%Finally, note that rotating $M$ by 180 degrees and removing the extra rows and columns added at the start results again in the original shape $F$
%and that the constraints that the fillings of $Q_A$ and $Q_B$ impose on the rest of the diagram correspond to the constraints
%of fillings in $\F^{NE}_{out}(F, l, \bm{r}, \bm{c}, \bm{a}, \bm{b})$.
%
\end{proof}

\section{Proof of Theorem \ref{thm_main}}

We start by using the growth diagram construction of Section \ref{sec3} to prove an additional useful bijection
between sets of constrained fillings described in Section \ref{sec4}.

\begin{lemma}\label{lemma_flip}
Let $F$ be a $NW$ Ferrers diagram with at least $n$ longest columns and at least $m$ longest rows, 
$l \geq 1$ an integer, $\bm{r}$ and $\bm{c}$ sequences prescribing nonempty
rows and columns of $F$, and $\bm{a} = (a_1, \ldots, a_n)$ and $\bm{b} = (b_1, \ldots, b_m)$ weakly increasing
sequences of integers less than equal to $l$. Then there is a~bijection between the sets
$\F^{NE}_{out}(F, l, \bm{r}, \bm{c}, \bm{a}, \bm{b})$ and $\F^{SE}_{out}(F, l, \bm{r}, \bm{c}, \bm{a}, \bm{b})$.
\end{lemma}
\begin{proof}
Consider a filling of $F$ from $\F^{NE}_{out}(F, l, \bm{r}, \bm{c}, \bm{a}, \bm{b})$
and build its growth diagram, obtaining the corresponding oscillating tableau $(\lambda^0, \lambda^1, \ldots, \lambda^k)$.
The constraints $\bm{a}$ together with Theorem \ref{thm_greene} imply that for $1 \leq i \leq n$ the partition $\lambda^i$
has exactly$a_i$ parts. Similarly the constraints $\bm{b}$ imply that for $1 \leq j \leq m$ the partition $\lambda^{k-j}$
has exactly $b_j$ parts. Applying the transformation described in Theorem \ref{thm_krat_3} to the filling of $F$,
we obtain a filling of $\F^{SE}(F, l, \bm{r}, \bm{c})$. In addition, since the oscillating tableau corresponding
to this new filling is obtained by transposing the oscillating tableau for the original filling,
we get for $1 \leq i \leq n$ that $(\lambda^i)^T_1 = a_i$ and so the longest $SE$-chain in the rectangle consisting of $i$ leftmost columns
of $F$ has length $a_i$ by Theorem \ref{thm_greene}. Similarly we get for $1 \leq j \leq m$
that $(\lambda^{k-j})^T_1 = b_j$ and so the longest $SE$-chain in the rectangle consisting of top $j$ rows of $F$ has length
 $b_j$ by Theorem \ref{thm_greene}. Therefore the obtained filling belongs
to $\F^{SE}_{out}(F, l, \bm{r}, \bm{c}, \bm{a}, \bm{b})$.
\end{proof}

Finally we have all we need to prove the main result.

\begin{proof}[Proof of Theorem \ref{thm_main}]
Lemma \ref{lemma_key} implies that it is enough to prove the theorem for simple skew diagrams. Let
$$S = F_1 |^v G_1 |^h F_2 |^v \cdots |^h F_n |^v G_n$$
be the simple representation of $S$.  The main idea of the proof
is now to apply Lemma \ref{lemma_constraints} to individual Ferrers diagrams in the simple representation of $S$ using constraints
constructed based on neighbouring diagrams.

Choose any filling from $\F^{NE}(S, l, \bm{r}, \bm{c})$.
Consider the diagram $F_i$ and let $\bm{r}^{F_i}$ and $\bm{c}^{F_i}$ be the sequences describing which rows and columns of $F_i$
currently contain a 1. The current filling of $F_i$ of course belongs to $\F^{NE}(F_i, l, \bm{r}^{F_i}, \bm{c}^{F_i})$. Let $n$ be the number
of columns of $F_i$ which are connected to a column of $G_{i-1}$ (if it exists, otherwise set $n$ to zero). Let $m$ be the number
of rows of $F_i$ which are connected to a row of $G_i$. We define the following rectangles:
\begin{itemize}
\item $A^{G_{i-1}}$ is the rectangle consisting of the $n$ rightmost columns of $G_{i-1}$,
\item $A^{F_i}$ is the rectangle consisting of the $n$ leftmost columns of $F_i$,
\item $B^{F_i}$ is the rectangle consisting of the top $m$ rows of $F_i$,
\item $B^{G_i}$ is the rectangle consisting of the bottom $m$ rows of $G_i$.
\end{itemize}
We define constraints $\bm{a} = (a_1, a_2, \ldots, a_n)$
and $\bm{b} = (b_1, b_2, \ldots, b_m)$ for fillings of $F_i$ as follows:
\begin{itemize}
\item if $x$ is the length of the longest $NE$-chain contained in $j$ rightmost columns of $A^{F_i}$, set $a_j$ to $x$,
\item if $y$ is the length of the longest $NE$-chain contained in the bottom $j$ rows of $B^{F_i}$, set $b_j$ to $y$.
\end{itemize}

From the way we defined $\bm{a}$ and $\bm{b}$ it is now clear that the filling of $F_i$
belongs to $\F^{NE}_{in}(F_i, l, \bm{r}^{F_i}, \bm{c}^{F_i}, \bm{a}, \bm{b})$. We now use the bijective map
of Lemma \ref{lemma_constraints} to transform the filling of $F_i$ into a filling
from $\F^{NE}_{out}(F_i, l, \bm{r}^{F_i}, \bm{c}^{F_i}, \bm{a}, \bm{b})$ and to this filling
we apply the bijective transformation of Lemma \ref{lemma_flip}, obtaining
a filling from $\F^{SE}_{out}(F_i, l, \bm{r}^{F_i}, \bm{c}^{F_i}, \bm{a}, \bm{b})$.

This way we transform the filling of every $F_i$ and every $G_i$ in the simple representation of $S$. The individual Ferrers
diagrams now avoid $SE$-chains longer than $l$, so it remains to prove that the fillings of rectangles connecting any two neighbouring diagrams
avoid them as well. Consider the diagrams $F_i$ and $G_i$. Any $SE$-chain longer than $l$ in $F_i |^v G_i$ would
have to be contained in the rectangle $R$ consisting of the rectangles $B^{F_i}$ and $B^{G_i}$.
Let $\bm{a}^{F_i}$ and $\bm{b}^{F_i}$ be the constraints for the filling of $F_i$ as constructed above
and let $\bm{a}^{G_i}$ and $\bm{b}^{G_i}$ be the same constraints for the filling of $G_i$. Suppose
for the sake of a contradiction that there is a $SE$-chain of length $l+1$ inside $R$ and that
$k$ 1's of the chain are contained in the top $j$ rows of $B^{F_i}$ and the remaining $l+1-k$ 1's of the chain
are contained in the bottom $m-j$ rows of $B^{G_i}$, thus $k \leq b^{F_i}_j$ and $l+1-k \leq b^{G_i}_{m-j}$,
giving $b^{F_i}_j + b^{G_i}_{m-j} \geq l+1$. On the other hand $b^{F_i}_j$ is the length of the longest $NE$-chain
contained in the bottom $j$ rows of $B^{F_i}$ in the original filling and $b^{G_i}_{m-j}$ is the length of the longest $NE$-chain
contained in the top $m-j$ rows of $B^{G_i}$ in the original filling and therefore $b^{F_i}_j + b^{G_i}_{m-j} \leq l$
and a contradiction is obtained. Therefore the resulting filling of $S$ belongs to $\F^{SE}(S, l, \bm{r}, \bm{c})$
and since it was obtained using bijective transformations, the proof is finished.
\end{proof}
