\chapter{Preliminaries}

\section{Polyominoes}

A \emph{cell} is a unit square whose vertices lie on lattice points of the $\Z^2$ plane.
We then define a \emph{polyomino} as a finite set of cells. When referring to \emph{coordinates} of a cell,
we will use the $\Z^2$ coordinates of its bottom left corner.
A polyomino is \emph{convex} if for any two of its cells in the same row or column the polyomino also contains every cell in between.
Two columns of a polyomino are \emph{comparable} if the set of row coordinates of one column is contained in the set of row coordinates
of the other. We say that a polyomino is \emph{intersection-free} if any two of its columns are comparable. Note that this is equivalent
to having any two of its rows comparable. A \emph{moon polyomino} is a convex, intersection-free polyomino. A polyomino
is \emph{top-justified} if the top ends of all of its columns are in the same row and it is \emph{left-justified}
if the left ends of all of its rows are in the same column. Similarly we define a \emph{bottom-justified} or a
\emph{right-justified} polyomino.
A moon polyomino is a \emph{Ferrers diagram} if it is top-justified or bottom justified and in addition either left-justified or right-justified.

\begin{figure}[h]
\centering
\subfloat[][non-convex] {
\begin{tikzpicture}[line cap=round,line join=round,>=triangle 45,x=0.75cm,y=0.75cm]
\clip(1.56,0.49) rectangle (5.37,4.25);
\draw (2,4)-- (2,1);
\draw (2,1)-- (5,1);
\draw (5,1)-- (5,4);
\draw (5,4)-- (4,4);
\draw (4,4)-- (4,2);
\draw (4,2)-- (3,2);
\draw (3,2)-- (3,4);
\draw (3,4)-- (2,4);
\draw (2,3)-- (3,3);
\draw (2,2)-- (3,2);
\draw (4,2)-- (5,2);
\draw (4,2)-- (4,1);
\draw (3,2)-- (3,1);
\draw (4,3)-- (5,3);
\end{tikzpicture}
}
\subfloat[][convex, not \\ intersection-free] {
\begin{tikzpicture}[line cap=round,line join=round,>=triangle 45,x=0.75cm,y=0.75cm]
\clip(5.9,0.53) rectangle (9.1,4.25);
\draw (6,3)-- (6,1);
\draw (7,4)-- (7,1);
\draw (8,4)-- (8,1);
\draw (9,4)-- (9,2);
\draw (9,2)-- (6,2);
\draw (8,1)-- (6,1);
\draw (6,3)-- (9,3);
\draw (9,4)-- (7,4);
\end{tikzpicture}
}
\subfloat[][moon polyomino] {
\begin{tikzpicture}[line cap=round,line join=round,>=triangle 45,x=0.75cm,y=0.75cm]
\clip(9.52,0.36) rectangle (13.5,4.21);
\draw (10,3)-- (13,3);
\draw (12,4)-- (12,1);
\draw (11,4)-- (11,1);
\draw (10,2)-- (13,2);
\draw (13,2)-- (13,3);
\draw (12,4)-- (11,4);
\draw (10,3)-- (10,2);
\draw (11,1)-- (12,1);
\end{tikzpicture}
}
\subfloat[][bottom-justified moon polyomino] {
\centering
\begin{tikzpicture}[line cap=round,line join=round,>=triangle 45,x=0.75cm,y=0.75cm]
\clip(13.9,0.4) rectangle (17.56,4.25);
\draw (14,1)-- (17,1);
\draw (15,1)-- (15,4);
\draw (16,1)-- (16,4);
\draw (14,1)-- (14,3);
\draw (14,3)-- (16,3);
\draw (15,4)-- (16,4);
\draw (14,2)-- (17,2);
\draw (17,1)-- (17,2);
\end{tikzpicture}
}
\caption{Examples of polyominoes}
\label{figure_examples}
\end{figure}


A \emph{$NW$ Ferrers diagram} (standing for \emph{northwest}) is top-justified and left-justified.
A \emph{$SE$ Ferrers diagram} (standing for \emph{southeast}) is bottom-justified and right-justified.
We can represent both kinds of Ferrers diagrams by sequences of characters $R$ (meaning a step right) and $U$ (meaning step up),
describing the right-up border or the up-right border of $NW$ Ferrers diagrams or $SE$ Ferrers diagrams respectively.
For example, the $NW$ Ferrers diagram in Figure \ref{fig_nw_se}(a) is represented by the sequence 
\textit{RRURURRURUU}
and the $SE$ Ferrers diagram in Figure \ref{fig_nw_se}(b) is represented by the sequence 
\textit{URUUURRURR}. We call the sequences of characters $R$ and $U$ the \emph{$R$-$U$ sequences} and if a Ferrers diagram $F$
is represented by an $R$-$U$ sequence $w$, we say that $F$ \emph{is of type $w$}.

\begin{figure}[h]
\centering
\subfloat[][$NW$ Ferrers diagram] {
\begin{tikzpicture}[line cap=round,line join=round,>=triangle 45,x=1.0cm,y=1.0cm]
\clip(1.6,0) rectangle (8.24,6.22);
\draw (2,1)-- (2,6);
\draw (2,6)-- (8,6);
\draw (2,1)-- (4,1);
\draw (4,1)-- (4,2);
\draw (4,2)-- (5,2);
\draw (5,2)-- (5,3);
\draw (8,6)-- (8,4);
\draw (8,4)-- (7,4);
\draw (7,4)-- (7,3);
\draw (7,3)-- (5,3);

\draw (3,6)-- (3,1);
\draw (4,6)-- (4,2);
\draw (5,6)-- (5,3);
\draw (6,6)-- (6,3);
\draw (7,6)-- (7,4);
\draw (8,5)-- (2,5);
\draw (7,4)-- (2,4);
\draw (5,3)-- (2,3);
\draw (4,2)-- (2,2);

\end{tikzpicture}
}
\subfloat[][$SE$ Ferrers diagram] {
\begin{tikzpicture}[line cap=round,line join=round,>=triangle 45,x=1.0cm,y=1.0cm]
\clip(10.66,0) rectangle (16.45,6.9);
\draw (11,1)-- (16,1);
\draw (16,1)-- (16,6);
\draw (11,1)-- (11,2);
\draw (11,2)-- (12,2);
\draw (12,2)-- (12,5);
\draw (12,5)-- (14,5);
\draw (14,5)-- (14,6);
\draw (14,6)-- (16,6);
\draw (12,2)-- (16,2);
\draw (12,3)-- (16,3);
\draw (12,4)-- (16,4);
\draw (14,5)-- (16,5);
\draw (15,6)-- (15,1);
\draw (12,2)-- (12,1);
\draw (13,5)-- (13,1);
\draw (14,5)-- (14,1);
\end{tikzpicture}
}
\caption{Examples of Ferrers diagrams}
\label{fig_nw_se}
\end{figure}

Given a convex polyomino $P$, we define its \emph{height} $h(P)$ as the number of its nonempty rows. 
Similarly, we define its \emph{width} $w(P)$ as the number of nonempty columns of $P$.

Sometimes we will need to speak about the boundary of a convex polyomino, considered as a closed convex subset of the plane. We 
call the boundary of a convex polyomino $P$ the \emph{border} of $P$. If two adjacent sides of a cell of $\Z^2$ lie
on the border of $P$, the corner of the cell in which the two sides meet is a \emph{turn}
on the border. A line along the border between two adjacent turns of $M$ is a \emph{segment}. For example, the border
of the polyomino in Figure \ref{figure_examples}(b) consists of 8 turns and 8 segments.

\section{Partitions}
A \emph{partition} of length $n$ is a finite weakly decreasing sequence of positive integers $\lambda = (\lambda_1, \lambda_2, \ldots, \lambda_n)$. 
Each of the numbers $\lambda_i$ is called a \emph{part} of $\lambda$ and the \emph{size} of $\lambda$ is the sum of its parts.
Such a partition $\lambda$ also represents a $NW$ Ferrers diagram with column heights $\lambda_1, \ldots, \lambda_n$.
For example, the diagram in Figure \ref{fig_nw_se}(a) is represented by a partition $(5, 5, 4, 3, 3, 2)$. 
A partition $\mu = (\mu_1, \ldots, \mu_m)$ is \emph{contained} in $\lambda = (\lambda_1, \ldots, \lambda_n)$ if $m \leq n$
and $\mu_i \leq \lambda_i$ for $1 \leq i \leq m$.
The \emph{union} of partitions $\mu = (\mu_1, \ldots, \mu_m)$ and $\lambda = (\lambda_1, \ldots, \lambda_n)$ is
the partition $\nu = (\nu_1, \ldots, \nu_k)$ such that $k = \max(m, n)$ and $\nu_i = max(\mu_i, \lambda_i)$.
Finally, the \emph{transpose} of a partition $\lambda = (\lambda_1, \ldots, \lambda_n)$ is the partition
$\lambda^T = (\lambda^T_1, \ldots, \lambda^T_t)$ such that $t = \lambda_1$ and $\lambda^T_i$ is equal
to the number of parts of $\lambda$ greater than or equal to $i$. Notice that transposing
a partition is equivalent to reflecting the associated Ferrers diagram over the $y = x$ line.

\section{Skew diagrams}
A skew diagram is a $NW$ Ferrers diagram with a smaller $NW$ Ferrers diagram cut out of its north-west corner. This is represented
by a pair of partitions $(\lambda, \mu)$ such that $\mu$ is contained in $\lambda$.

\begin{figure}[h]
\centering
\begin{tikzpicture}[line cap=round,line join=round,>=triangle 45,x=1.0cm,y=1.0cm]
\clip(-4.15,-1.14) rectangle (1.12,3.15);
\draw (-4,-1)-- (-2,-1);
\draw (-2,-1)-- (-2,0);
\draw (-2,0)-- (0,0);
\draw (0,0)-- (0,2);
\draw (0,2)-- (1,2);
\draw (1,2)-- (1,3);
\draw (1,3)-- (-2,3);
\draw (-2,3)-- (-2,2);
\draw (-2,2)-- (-3,2);
\draw (-3,2)-- (-3,1);
\draw (-3,1)-- (-4,1);
\draw (-4,1)-- (-4,-1);
\draw [dash pattern=on 2pt off 2pt] (-4,1)-- (-4,3);
\draw [dash pattern=on 2pt off 2pt] (-4,3)-- (-2,3);
\draw (-2,3)-- (-2,0);
\draw (-1,3)-- (-1,0);
\draw (0,3)-- (0,2);
\draw (0,2)-- (-2,2);
\draw (0,1)-- (-3,1);
\draw (-2,0)-- (-4,0);
\draw [dash pattern=on 2pt off 2pt] (-3,2)-- (-3,3);
\draw [dash pattern=on 2pt off 2pt] (-3,2)-- (-4,2);
\draw (-3,1)-- (-3,-1);
\end{tikzpicture}
\caption{The skew diagram $((4,4,3,3,1),(2,1))$}
\end{figure}

Let $S$ be a skew diagram and let $C$ be one of its columns. We define the operation of \emph{deleting} the column $C$ from $S$
as removing the cells of $C$ and shifting all cells of $S$ right of $C$ one cell to the left. Similarly we define
the operation of deleting a row $R$ from $S$ as removing the cells of $R$ from $S$ and shifting all cells of $S$ below $R$
one cell upward.
A skew diagram $S$ is a \emph{subdiagram} of a skew diagram $T$ or is \emph{contained} in $T$ if $S$ can be created from $T$
by deleting some rows and columns. If $S$ is not containted in $T$, we say that $T$ \emph{avoids} $S$. Note that
both the class of skew diagrams and the class of Ferrers diagrams are closed under the operations of deleting a row or a column,
i.e. deleting a row or a column from a skew diagram always results in a skew diagram and in addition, if the diagram
is a Ferrers diagram, it remains a Ferrers diagram.

\begin{figure}[h]
\centering
\subfloat[][]{
\begin{tikzpicture}[line cap=round,line join=round,>=triangle 45,x=0.75cm,y=0.75cm]
\clip(-4.15,-1.14) rectangle (1.12,3.15);
\draw (-4,-1)-- (-2,-1);
\draw (-2,-1)-- (-2,0);
\draw (-2,0)-- (0,0);
\draw (0,0)-- (0,2);
\draw (0,2)-- (1,2);
\draw (1,2)-- (1,3);
\draw (1,3)-- (-2,3);
\draw (-2,3)-- (-2,2);
\draw (-2,2)-- (-3,2);
\draw (-3,2)-- (-3,1);
\draw (-3,1)-- (-4,1);
\draw (-4,1)-- (-4,-1);
\draw (-2,3)-- (-2,0);
\draw (-1,3)-- (-1,0);
\draw (0,3)-- (0,2);
\draw (0,2)-- (-2,2);
\draw (0,1)-- (-3,1);
\draw (-2,0)-- (-4,0);
\draw (-3,1)-- (-3,-1);
\draw [line width=1.6pt] (-3,-1) -- (-2,-1);
\draw [line width=1.6pt] (-2,-1) -- (-2,2);
\draw [line width=1.6pt] (-2,2) -- (-3,2);
\draw [line width=1.6pt] (-3,2) -- (-3,-1);
\end{tikzpicture}
}
\subfloat[][]{
\begin{tikzpicture}[line cap=round,line join=round,>=triangle 45,x=0.75cm,y=0.75cm]
\clip(-4.15,-1.14) rectangle (1.12,3.15);
\draw (-3,-1)-- (-2,-1);
\draw (-2,-1)-- (-2,0);
\draw (-2,0)-- (0,0);
\draw (0,0)-- (0,2);
\draw (0,2)-- (1,2);
\draw (1,2)-- (1,3);
\draw (1,3)-- (-2,3);
\draw (-2,3)-- (-2,2);
\draw (-3,1)-- (-3,1);
\draw (-3,1)-- (-3,-1);
\draw (-2,3)-- (-2,0);
\draw (-1,3)-- (-1,0);
\draw (0,3)-- (0,2);
\draw (0,2)-- (-2,2);
\draw (0,1)-- (-3,1);
\draw (-2,0)-- (-3,0);
\draw (-3,1)-- (-3,-1);
\end{tikzpicture}
}
\caption{Deleting a column from a skew diagram}
\label{figure_del_col}
\end{figure}

A skew diagram is clearly bounded by two paths leading from its lower left and to its upper right corner
and consisting only of steps up and to the right. We shall call them the \emph{upper border}
and the \emph{lower border} of the diagram.

For convenience, we will only consider \emph{connected} skew diagrams, which satisfy the additional condition that their
upper and lower borders only meet in the lower left and the upper right corner and nowhere in between. All of the presented 
results about connected skew diagrams can be extended to general skew diagrams without any effort.

\section{Fillings of polyominoes}

The main theme of this thesis is filling cells of polyominoes with integers and then counting fillings having certain properties.
A \emph{0-1-filling} of a polyomino is an assignment of either a 0 or a 1 to each cell of the polyomino.
A \emph{sparse filling} is a 0-1-filling, such that there is at most one 1 in each row and in each column. 
In figures, we will often represent 0's in such fillings by empty cells and 1's by crosses. In addition,
if a subset of a polyomino contains only 0's, we shall say that it is \emph{empty}, otherwise it is \emph{nonempty}.

\begin{figure}
\centering
\subfloat[][A sparse filling] {
\begin{tikzpicture}[line cap=round,line join=round,>=triangle 45,x=0.75cm,y=0.75cm]
\clip(-0.21,-1.19) rectangle (6.12,5.2);
\draw (1,4)-- (1,0);
\draw (1,0)-- (5,0);
\draw (5,0)-- (5,4);
\draw (5,4)-- (1,4);
\draw (2,0)-- (2,5);
\draw (3,0)-- (3,5);
\draw (4,0)-- (4,5);
\draw (2,5)-- (5,5);
\draw (5,5)-- (5,4);
\draw (3,0)-- (3,-1);
\draw (3,-1)-- (6,-1);
\draw (6,-1)-- (6,2);
\draw (6,2)-- (5,2);
\draw (5,1)-- (0,1);
\draw (5,2)-- (0,2);
\draw (5,3)-- (0,3);
\draw (0,1)-- (0,4);
\draw (0,4)-- (1,4);
\draw (4,0)-- (4,-1);
\draw (5,0)-- (5,-1);
\draw (5,0)-- (6,0);
\draw (5,1)-- (6,1);
\draw (0.5,3.5) node {$\times$};
\draw (2.5,2.5) node {$\times$};
\draw (4.5,1.5) node {$\times$};
\draw (1.5,0.5) node {$\times$};
\draw (3.5,4.5) node {$\times$};
\draw (5.5,-0.5) node {$\times$};
\end{tikzpicture}
}
\subfloat[][A non-sparse filling] {
\begin{tikzpicture}[line cap=round,line join=round,>=triangle 45,x=0.75cm,y=0.75cm]
\clip(-0.21,-1.19) rectangle (6.12,5.2);
\draw (1,4)-- (1,0);
\draw (1,0)-- (5,0);
\draw (5,0)-- (5,4);
\draw (5,4)-- (1,4);
\draw (2,0)-- (2,5);
\draw (3,0)-- (3,5);
\draw (4,0)-- (4,5);
\draw (2,5)-- (5,5);
\draw (5,5)-- (5,4);
\draw (3,0)-- (3,-1);
\draw (3,-1)-- (6,-1);
\draw (6,-1)-- (6,2);
\draw (6,2)-- (5,2);
\draw (5,1)-- (0,1);
\draw (5,2)-- (0,2);
\draw (5,3)-- (0,3);
\draw (0,1)-- (0,4);
\draw (0,4)-- (1,4);
\draw (4,0)-- (4,-1);
\draw (5,0)-- (5,-1);
\draw (5,0)-- (6,0);
\draw (5,1)-- (6,1);
\draw (1.5,0.5) node {$\times$};
\draw (1.5,2.5) node {$\times$};
\draw (2.5,2.5) node {$\times$};
\draw (3.5,2.5) node {$\times$};
\draw (3.5,3.5) node {$\times$};
\draw (4.5,4.5) node {$\times$};
\end{tikzpicture}
}
\caption{Examples of 0-1-fillings}
\label{figure_fillings}
\end{figure}

Let $P$ be a polyomino filled with a 0-1-filling and let $c_1, c_2, \ldots, c_l$ be $l$ of its cells filled with a 1.
These cells form a $NE$\emph{-chain (pronounced "northeast chain") of length} $l$ in $P$ if for $1 < i \leq l$ each $c_i$ is 
strictly northeast of $c_{i-1}$ and the smallest rectangle containing all $l$
cells is entirely contained in $P$. Similarly the cells form a $SE$\emph{-chain (pronounced "southeast chain") of length} $l$, if
for $1 < i \leq l$ each $c_i$ is strictly southeast of $c_{i-1}$ and again the smallest rectangle containing all the cells is entirely
contained in $P$. We then say that $P$ \emph{contains} a $NE$-chain (or $SE$-chain) of length $l$ if there is at least one 
occurence of such a chain in $P$. Otherwise we say that $P$ \emph{avoids} a $NE$-chain (or $SE$-chain) of length $l$.
For example, in Figure \ref{figure_chains}(a) the smallest rectangle containing the three nonzero entries are contained in the 
polyomino, so the 1's do form a $SE$-chain of length 3. However, in Figure \ref{figure_chains}(b) the northwest corner
of the smallest rectangle containing the four nonempty cells is not a part of the polyomino, so this is not a $NE$-chain of length 4.
However both the lower three 1's and the upper three 1's form a $NE$-chain of length 3.

\begin{figure}
\centering
\subfloat[][A $SE$-chain of length 3] {
\begin{tikzpicture}[line cap=round,line join=round,>=triangle 45,x=0.75cm,y=0.75cm]
\clip(-0.21,-1.19) rectangle (6.12,5.2);
\draw (1,4)-- (1,0);
\draw (1,0)-- (5,0);
\draw (5,0)-- (5,4);
\draw (5,4)-- (1,4);
\draw (2,0)-- (2,5);
\draw (3,0)-- (3,5);
\draw (4,0)-- (4,5);
\draw (2,5)-- (5,5);
\draw (5,5)-- (5,4);
\draw (3,0)-- (3,-1);
\draw (3,-1)-- (6,-1);
\draw (6,-1)-- (6,2);
\draw (6,2)-- (5,2);
\draw (5,1)-- (0,1);
\draw (5,2)-- (0,2);
\draw (5,3)-- (0,3);
\draw (0,1)-- (0,4);
\draw (0,4)-- (1,4);
\draw (4,0)-- (4,-1);
\draw (5,0)-- (5,-1);
\draw (5,0)-- (6,0);
\draw (5,1)-- (6,1);
\draw (0.5,3.5) node {$\times$};
\draw (2.5,2.5) node {$\times$};
\draw (4.5,1.5) node {$\times$};
\end{tikzpicture}
}
\subfloat[][Not a $NE$-chain of length 4] {
\begin{tikzpicture}[line cap=round,line join=round,>=triangle 45,x=0.75cm,y=0.75cm]
\clip(-0.21,-1.19) rectangle (6.12,5.2);
\draw (1,4)-- (1,0);
\draw (1,0)-- (5,0);
\draw (5,0)-- (5,4);
\draw (5,4)-- (1,4);
\draw (2,0)-- (2,5);
\draw (3,0)-- (3,5);
\draw (4,0)-- (4,5);
\draw (2,5)-- (5,5);
\draw (5,5)-- (5,4);
\draw (3,0)-- (3,-1);
\draw (3,-1)-- (6,-1);
\draw (6,-1)-- (6,2);
\draw (6,2)-- (5,2);
\draw (5,1)-- (0,1);
\draw (5,2)-- (0,2);
\draw (5,3)-- (0,3);
\draw (0,1)-- (0,4);
\draw (0,4)-- (1,4);
\draw (4,0)-- (4,-1);
\draw (5,0)-- (5,-1);
\draw (5,0)-- (6,0);
\draw (5,1)-- (6,1);
\draw (1.5,0.5) node {$\times$};
\draw (2.5,2.5) node {$\times$};
\draw (3.5,3.5) node {$\times$};
\draw (4.5,4.5) node {$\times$};
\end{tikzpicture}
}
\caption{Examples of chains in a polyomino with a 0-1-filling}
\label{figure_chains}
\end{figure}

In the following chapter we deal with sparse fillings of skew diagrams avoiding long $NE$-chains and $SE$-chains. To describe
the sets of fillings in question easily, we will use the notation introduced by Rubey \cite{Rubey11}.

\begin{defn}[{\cite[Definition 5.2]{Rubey11}}]\label{def_rubey}
Let $P$ be a moon polyomino or a connected skew diagram, 
$l$ a positive integer and $\bm{r}$, $\bm{c}$ sequences of 0's and 1's of lengths $h(P)$ and $w(P)$ respectively.
Then $\F^{NE}(P, l, \bm{r}, \bm{c})$ is the set of all sparse fillings of $P$ such that
\begin{itemize}
\item there is a 1 in the $i$-th row of $P$ if and only if $r_i = 1$,
\item there is a 1 in the $i$-th column of $P$ if and only if $c_i = 1$,
\item the length of the longest $NE$-chain in the filling is equal to $l$.
\end{itemize}
\end{defn}

Similarly we define $\F^{SE}(P, l, \bm{r}, \bm{c})$ as the set of all sparse fillings of $P$ with the length of the longest $SE$-chain
equal to $l$ and having 1's exactly in the rows and columns prescribed by $\bm{r}$ and $\bm{c}$. Note that since we are dealing with sparse fillings,
the number of nonempty rows and columns prescribed by $\bm{r}$ and $\bm{c}$ must be the same for any fillings to exist.

