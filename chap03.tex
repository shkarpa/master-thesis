\chapter{General 0-1-fillings avoiding a chain of length 2}

Given any skew diagram $S$, it is known that there are at least as many sparse 0-1-fillings of $S$ avoiding an $NE$-chain
of length 2 as there are sparse 0-1-fillings of $S$ avoiding an $SE$-chain of length 2. In this chapter,
we will extend this result to general 0-1-fillings.

Recall the skew diagram $S^{forb}$ defined in the previous chapter. We denote by $S^{forb}(132)$ the diagram $S^{forb}$
associated with a sparse 0-1-filling of a 132 pattern, as shown in the Figure \ref{figure_sforb_132}.
We then say that a filling of a skew diagram $S$ \emph{contains} $S^{forb}(132)$ if we can obtain $S^{forb}(132)$ by
removing some columns and rows of $S$ and replacing some entries 1 by entries 0. Otherwise the filling of $S$ \emph{avoids} $S^{forb}(132)$.

\begin{figure}[h]
\centering
\begin{tikzpicture}[line cap=round,line join=round,>=triangle 45,x=0.75cm,y=0.75cm]
\clip(5.9,0.53) rectangle (9.1,4.25);
\draw (6,3)-- (6,1);
\draw (7,4)-- (7,1);
\draw (8,4)-- (8,1);
\draw (9,4)-- (9,2);
\draw (9,2)-- (6,2);
\draw (8,1)-- (6,1);
\draw (6,3)-- (9,3);
\draw (9,4)-- (7,4);
\draw (6.5, 1.5) node {$\times$};
\draw (8.5, 2.5) node {$\times$};
\draw (7.5, 3.5) node {$\times$};
\end{tikzpicture}
\caption{$S^{forb}(132)$}
\label{figure_sforb_132}
\end{figure}

The following theorem is a simple consequence of a known result due to Jelínek \cite[Lemmas 29 and 30]{Jelinek08}.
\begin{thm}\label{thm_1}
Let $S$ be a skew diagram. Then the number of sparse 0-1-fillings of $S$ avoiding an $NE$-chain of length 2 and $S^{forb}(132)$
is equal to the number of sparse 0-1-fillings of $S$ avoiding an $SE$-chain of length 2.
\end{thm}

The goal of this chapter is to prove the following stronger claim.
\begin{thm} \label{thm_2}
Let $S$ be a skew diagram. Then there is a bijection between general 0-1-fillings of $S$ avoiding a $NE$-chain of length 2 and $S^{forb}(132)$
and general 0-1-fillings of $S$ avoiding a $SE$-chain of length 2.
\end{thm}

Given a skew diagram $S$ with a total of $N$ cells, we assign labels $c_1, c_2, \ldots, c_N$ to every cell starting from the lower left
corner, iterating over rows from the bottom to the top of $S$ and labeling cells in a row from left to right, as indicated in Figure 
\ref{figure_cell_labels}.

We associate with every cell $c$ a three-part piecewise linear curve $l(c)$ consisting of the ray going from the upper-right corner of $c$
to the left, the right border of $c$ and the ray going from the lower-right corner of $c$ to the right. The curve $l(c_i)$
of a cell $c_i$ of a skew diagram $S$ clearly divides it into two parts, as illustrated in Figure \ref{figure_cell_labels}.

\begin{figure}[h]
\centering
\begin{tikzpicture}[line cap=round,line join=round,>=triangle 45,x=1.0cm,y=1.0cm]
\clip(-6,-1.14) rectangle (3,3.15);
\draw (-4,-1)-- (-2,-1);
\draw (-2,-1)-- (-2,0);
\draw (-2,0)-- (0,0);
\draw (0,0)-- (0,2);
\draw (0,2)-- (1,2);
\draw (1,2)-- (1,3);
\draw (1,3)-- (-2,3);
\draw (-2,3)-- (-2,2);
\draw (-2,2)-- (-3,2);
\draw (-3,2)-- (-3,1);
\draw (-3,1)-- (-4,1);
\draw (-4,1)-- (-4,-1);
\draw (-2,3)-- (-2,0);
\draw (-1,3)-- (-1,0);
\draw (0,3)-- (0,2);
\draw (0,2)-- (-2,2);
\draw (0,1)-- (-3,1);
\draw (-2,0)-- (-4,0);
\draw (-3,1)-- (-3,-1);
\draw (-3.5, -0.5) node {$c_1$};
\draw (-2.5, -0.5) node {$c_2$};
\draw (-3.5, 0.5) node {$c_3$};
\draw (-2.5, 0.5) node {$c_4$};
\draw (-1.5, 0.5) node {$c_5$};
\draw (-0.5, 0.5) node {$c_6$};
\draw (-2.5, 1.5) node {$c_7$};
\draw (-1.5, 1.5) node {$c_8$};
\draw (-0.5, 1.5) node {$c_9$};
\draw (-1.5, 2.5) node {$c_{10}$};
\draw (-0.5, 2.5) node {$c_{11}$};
\draw (0.5, 2.5) node {$c_{12}$};
\draw [line width=1.6pt] (-1,2) -- (-10,2);
\draw [line width=1.6pt] (-1,1) -- (10,1);
\draw [line width=1.6pt] (-1,2) -- (-1,1);
\end{tikzpicture}
\caption{A skew diagram with cell labels and the curve $l(c_8)$}
\label{figure_cell_labels}
\end{figure}

We continue by defining a special set of fillings for every cell of a skew diagram.
Let $N$ be the number of cells of a skew diagram $S$ and let $i$ be an integer between 1 and $N$. We define the set $\mathcal{G}_i(S)$
as the set of all 0-1-fillings of $S$ which satisfy the following conditions:
\begin{enumerate}[(a)]
\item there is no occurence of a $SE$-chain of length 2 with both 1's above $l(c_i)$,
\item there is no occurence of a $SE$-chain of length 2 with the upper 1 above $l(c_i)$ and the lower 1 below $l(c_i)$,
\item there is no occurence of a $SE$-chain of length 2 with the upper 1 in the same row as $c_i$ and the lower 1 strictly right of $c_i$,
\item there is no occurence of a $NE$-chain of length 2 entirely below $l(c_i)$,
\item there is no occurence of $S^{forb}(132)$ entirely below $l(c_i)$.
\end{enumerate}

\begin{figure}[h]
\centering
\subfloat[][] {
\begin{tikzpicture}[line cap=round,line join=round,>=triangle 45,x=0.5cm,y=0.5cm]
\clip(-6,-1.14) rectangle (3,3.15);
\draw (-4,-1)-- (-2,-1);
\draw (-2,-1)-- (-2,0);
\draw (-2,0)-- (0,0);
\draw (0,0)-- (0,2);
\draw (0,2)-- (1,2);
\draw (1,2)-- (1,3);
\draw (1,3)-- (-2,3);
\draw (-2,3)-- (-2,2);
\draw (-2,2)-- (-3,2);
\draw (-3,2)-- (-3,1);
\draw (-3,1)-- (-4,1);
\draw (-4,1)-- (-4,-1);
\draw (-2,3)-- (-2,0);
\draw (-1,3)-- (-1,0);
\draw (0,3)-- (0,2);
\draw (0,2)-- (-2,2);
\draw (0,1)-- (-3,1);
\draw (-2,0)-- (-4,0);
\draw (-3,1)-- (-3,-1);
\draw [line width=1.6pt] (-1,2) -- (-10,2);
\draw [line width=1.6pt] (-1,1) -- (10,1);
\draw [line width=1.6pt] (-1,2) -- (-1,1);
\draw (-1.5, 2.5) node {$\times$};
\draw (-0.5, 1.5) node {$\times$};
\end{tikzpicture}
}
\subfloat[][] {
\begin{tikzpicture}[line cap=round,line join=round,>=triangle 45,x=0.5cm,y=0.5cm]
\clip(-6,-1.14) rectangle (3,3.15);
\draw (-4,-1)-- (-2,-1);
\draw (-2,-1)-- (-2,0);
\draw (-2,0)-- (0,0);
\draw (0,0)-- (0,2);
\draw (0,2)-- (1,2);
\draw (1,2)-- (1,3);
\draw (1,3)-- (-2,3);
\draw (-2,3)-- (-2,2);
\draw (-2,2)-- (-3,2);
\draw (-3,2)-- (-3,1);
\draw (-3,1)-- (-4,1);
\draw (-4,1)-- (-4,-1);
\draw (-2,3)-- (-2,0);
\draw (-1,3)-- (-1,0);
\draw (0,3)-- (0,2);
\draw (0,2)-- (-2,2);
\draw (0,1)-- (-3,1);
\draw (-2,0)-- (-4,0);
\draw (-3,1)-- (-3,-1);
\draw [line width=1.6pt] (-1,2) -- (-10,2);
\draw [line width=1.6pt] (-1,1) -- (10,1);
\draw [line width=1.6pt] (-1,2) -- (-1,1);
\draw (-1.5, 2.5) node {$\times$};
\draw (-0.5, 0.5) node {$\times$};
\end{tikzpicture}
}
\subfloat[][] {
\begin{tikzpicture}[line cap=round,line join=round,>=triangle 45,x=0.5cm,y=0.5cm]
\clip(-6,-1.14) rectangle (3,3.15);
\draw (-4,-1)-- (-2,-1);
\draw (-2,-1)-- (-2,0);
\draw (-2,0)-- (0,0);
\draw (0,0)-- (0,2);
\draw (0,2)-- (1,2);
\draw (1,2)-- (1,3);
\draw (1,3)-- (-2,3);
\draw (-2,3)-- (-2,2);
\draw (-2,2)-- (-3,2);
\draw (-3,2)-- (-3,1);
\draw (-3,1)-- (-4,1);
\draw (-4,1)-- (-4,-1);
\draw (-2,3)-- (-2,0);
\draw (-1,3)-- (-1,0);
\draw (0,3)-- (0,2);
\draw (0,2)-- (-2,2);
\draw (0,1)-- (-3,1);
\draw (-2,0)-- (-4,0);
\draw (-3,1)-- (-3,-1);
\draw [line width=1.6pt] (-1,2) -- (-10,2);
\draw [line width=1.6pt] (-1,1) -- (10,1);
\draw [line width=1.6pt] (-1,2) -- (-1,1);
\draw (-2.5, 1.5) node {$\times$};
\draw (-0.5, 0.5) node {$\times$};
\end{tikzpicture}
}

\subfloat[][] {
\begin{tikzpicture}[line cap=round,line join=round,>=triangle 45,x=0.5cm,y=0.5cm]
\clip(-6,-1.14) rectangle (3,3.15);
\draw (-4,-1)-- (-2,-1);
\draw (-2,-1)-- (-2,0);
\draw (-2,0)-- (0,0);
\draw (0,0)-- (0,2);
\draw (0,2)-- (1,2);
\draw (1,2)-- (1,3);
\draw (1,3)-- (-2,3);
\draw (-2,3)-- (-2,2);
\draw (-2,2)-- (-3,2);
\draw (-3,2)-- (-3,1);
\draw (-3,1)-- (-4,1);
\draw (-4,1)-- (-4,-1);
\draw (-2,3)-- (-2,0);
\draw (-1,3)-- (-1,0);
\draw (0,3)-- (0,2);
\draw (0,2)-- (-2,2);
\draw (0,1)-- (-3,1);
\draw (-2,0)-- (-4,0);
\draw (-3,1)-- (-3,-1);
\draw [line width=1.6pt] (-1,2) -- (-10,2);
\draw [line width=1.6pt] (-1,1) -- (10,1);
\draw [line width=1.6pt] (-1,2) -- (-1,1);
\draw (-3.5, -0.5) node {$\times$};
\draw (-2.5, 0.5) node {$\times$};
\end{tikzpicture}
}
\subfloat[][] {
\begin{tikzpicture}[line cap=round,line join=round,>=triangle 45,x=0.5cm,y=0.5cm]
\clip(-6,-1.14) rectangle (3,3.15);
\draw (-4,-1)-- (-2,-1);
\draw (-2,-1)-- (-2,0);
\draw (-2,0)-- (0,0);
\draw (0,0)-- (0,2);
\draw (0,2)-- (1,2);
\draw (1,2)-- (1,3);
\draw (1,3)-- (-2,3);
\draw (-2,3)-- (-2,2);
\draw (-2,2)-- (-3,2);
\draw (-3,2)-- (-3,1);
\draw (-3,1)-- (-4,1);
\draw (-4,1)-- (-4,-1);
\draw (-2,3)-- (-2,0);
\draw (-1,3)-- (-1,0);
\draw (0,3)-- (0,2);
\draw (0,2)-- (-2,2);
\draw (0,1)-- (-3,1);
\draw (-2,0)-- (-4,0);
\draw (-3,1)-- (-3,-1);
\draw [line width=1.6pt] (-1,2) -- (-10,2);
\draw [line width=1.6pt] (-1,1) -- (10,1);
\draw [line width=1.6pt] (-1,2) -- (-1,1);
\draw (-3.5, -0.5) node {$\times$};
\draw (-2.5, 1.5) node {$\times$};
\draw (-1.5, 0.5) node {$\times$};
\end{tikzpicture}
}
\caption{Forbidden patterns of $\mathcal{G}_8(S)$}
\end{figure}

Note that $\mathcal{G}_1(S)$ is the set of all 0-1-fillings of $S$ avoiding a $SE$-chain of length 2 and $\mathcal{G}_N(S)$
is the set of all 0-1-fillings of $S$ avoiding a $NE$-chain of length 2 and $S^{forb}(132)$. Therefore,
 to prove Theorem \ref{thm_2} it is enough to construct 
 a chain of bijections between consecutive sets $\mathcal{G}_i(S)$ and $\mathcal{G}_{i+1}(S)$, which is done by the following lemma.

\begin{lemma}\label{lemma_gi}
Let $S$ be a skew diagram with $N$ cells and let $1 \leq i < N$. Then there is a bijection between fillings in $\mathcal{G}_i(S)$
and $\mathcal{G}_{i+1}(S)$.
\end{lemma}
\begin{proof}
First, consider the case where the cells $c_i$ and $c_{i+1}$ are not in the same row, i.e. $c_{i+1}$ is the first cell of a new row
and $c_i$ is the last cell of the previous row. In this case we will show that the sets $\mathcal{G}_i(S)$ and $\mathcal{G}_{i+1}(S)$
are identical.

Choose any filling of $S$ from $\mathcal{G}_i(S)$. Suppose that the selected filling contains an occurence of 
a pattern forbidden in $\mathcal{G}_{i+1}(S)$. Since $l(c_i)$ and $l(c_{i+1})$ create the same division of $S$ into two parts
except for the cell $c_{i+1}(S)$, the occurence of a forbidden pattern must use the cell $c_{i+1}$ filled with a 1, otherwise it would also be
forbidden in $\mathcal{G}_i(S)$. Since $c_{i+1}$ starts a row, it cannot be used to break the conditions (a), (b) or (d) of $\mathcal{G}_{i+1}(S)$.
Also since the rest of the row starting with $c_{i+1}$ is above $l(c_{i+1})$, the condition (e) cannot be broken either. Therefore the supposed
occurence of a forbidden pattern must break the condition (c) of $\mathcal{G}_{i+1}(S)$. However, such an occurence would break
the condition (b) of $\mathcal{G}_i(S)$, which is a contradiction, and so the chosen filling is also in $\mathcal{G}_{i+1}(S)$.

Now choose any filling of $S$ from $\mathcal{G}_{i+1}(S)$ and again suppose that it breaks at least one condition of $\mathcal{G}_i(S)$.
Using similar arguments as before we can show that this must be the condition (b) and a 1 in the cell $c_{i+1}$ is used as the upper
1 of the $SE$-chain. Then this $SE$-chain breaks the condition (c) of $\mathcal{G}_{i+1}(S)$ and the contradiction is reached again,
showing that indeed $\mathcal{G}_i = \mathcal{G}_{i+1}$.


The second and more interesting case is when the cells $c_i$ and $c_{i+1}$ are adjacent cells in a row. We will divide each of the 
sets $\mathcal{G}_i{S}$ and $\mathcal{G}_{i+1}(S)$ into two disjoint parts and construct bijections between corresponding pairs.
Let $R$ be the row of all cells strictly left of $c_{i+1}$, let $C$ be the column of all cells strictly below $c_{i+1}$
and let $A$ be the rectangle consisting of all cells that are strictly below $R$ and left of $C$. Note that $A$, $R$, $C$ and $c_{i+1}$
form the rectangle $M$ which is the maximal rectangle contained in $S$ with $c_{i+1}$ as its northeast corner. 
Therefore, in any occurence of a $NE$-chain with the upper 1 in the cell $c_{i+1}$, the lower 1 is inside the rectangle $A$.

\begin{figure}[h]
\centering
\begin{tikzpicture}[line cap=round,line join=round,>=triangle 45,x=0.75cm,y=0.75cm]
\clip(0.67,0.63) rectangle (14.32,12.34);
\draw (1,1)-- (5,1);
\draw (5,1)-- (5,2);
\draw (5,2)-- (8,2);
\draw (8,2)-- (8,4);
\draw (8,4)-- (11,4);
\draw (11,4)-- (11,6);
\draw (11,6)-- (13,6);
\draw (13,6)-- (13,9);
\draw (13,9)-- (14,9);
\draw (14,9)-- (14,12);
\draw (1,1)-- (1,4);
\draw (1,4)-- (2,4);
\draw (2,4)-- (2,8);
\draw (2,8)-- (4,8);
\draw (4,8)-- (4,11);
\draw (4,11)-- (8,11);
\draw (8,11)-- (8,12);
\draw (8,12)-- (14,12);
\draw (2,8)-- (9,8);
\draw [dash pattern=on 2pt off 2pt] (9,8)-- (9,4);
\draw [dash pattern=on 2pt off 2pt] (9,4)-- (2,4);
\draw [dash pattern=on 2pt off 2pt] (8,4)-- (8,8);
\draw [dash pattern=on 2pt off 2pt] (2,7)-- (8,7);
\draw (5, 7.5) node {$R$};
\draw (5, 5.5) node {$A$};
\draw (8.5, 5.5) node {$C$};
\draw (8.5, 7.5) node {$c_{i+1}$};
\draw [line width=1.6pt] (2,8) -- (9,8);
\draw [line width=1.6pt] (9,8) -- (9,7);
\draw [line width=1.6pt] (8,8) -- (8,7);
\draw [line width=1.6pt] (8,7) -- (13,7);
\end{tikzpicture}
\caption{The maximal rectangle $M$ with $c_{i+1}$ in the northeast corner}
\label{figure_max_rec}
\end{figure}


We divide $\mathcal{G}_i(S)$ into two disjoint sets as follows:
\begin{itemize}
\item $\mathcal{G}_i^1(S)$ contains the fillings in which there is either no 1 in $c_{i+1}$ or no 1 inside $A$,
\item $\mathcal{G}_i^2(S)$ contains the fillings with a 1 in $c_{i+1}$ and at least one 1 inside $A$.
\end{itemize}

We divide $\mathcal{G}_{i+1}(S)$ into five disjoint sets as follows:
\begin{itemize}
\item $\mathcal{G}_{i+1}^1(S)$ contains the fillings in which there is either no 1 inside $R$ or no 1 inside $C$,
\item $\mathcal{G}_{i+1}^2(S)$ contains the fillings in which both $R$ and $C$ are nonempty. 
\end{itemize}

First of all we show that, similarly as in the first part of the proof, the sets $\mathcal{G}_i^1(S)$ and $\mathcal{G}_{i+1}^1(S)$
in fact contain the same fillings, so we may use the identity map between them. Choose a filling from $\mathcal{G}_i^1(S)$.
Clearly, if $c_i$ and $c_{i+1}$ are adjacent cells, a filling that satisfies the conditions of $\mathcal{G}_i(S)$ 
also satisfies the conditions (a), (b) and (c) of $\mathcal{G}_{i+1}(S)$. The condition (d) can only be broken by a $NE$-chain of 
length 2 with the upper 1 in the cell $c_{i+1}$, but there is no 1 in that cell in the fillings of $\mathcal{G}_i^1(S)$.
Finally, if there is an occurence of $S^{forb}(132)$ in the filling that breaks the condition (e) of $\mathcal{G}_{i+1}(S)$
but not of $\mathcal{G}_i(S)$, it must be the case that the upper right corner of the occurence is the cell $c_{i+1}$ and therefore 
there is a 1 in $R$ and a 1 in $C$ and so the condition (c) of $\mathcal{G}_i(S)$ is broken. Therefore the selected filling
is in $\mathcal{G}_{i+1}^1$. By reverting the arguments we easily get also the opposite inclusion and so $\mathcal{G}_i^1(S) =
\mathcal{G}_{i+1}^1(S)$.

For the fillings of $\mathcal{G}^2_i(S)$ 
we perform the following transformation $f$ into a filling of $\mathcal{G}^2_{i+1}(S)$:
\begin{enumerate}
\item Label all nonempty columns of $A$ from left to right as $C_1, C_2, \ldots C_k$.
\item If $C$ is nonempty, then $R$ is empty and we move the 1 from the cell $c_{i+1}$ to the cell of $R$ above
the column $C_1$ and finish.
\item If $C$ is empty, then replace the filling of $C$ by the filling of $C_k$ and for $1 \leq j < k$ replace
the filling of $C_{j+1}$ by the filling of $C_j$. Replace the filling of $C_1$ by all zeros.
\item If there is a 1 in the cell of $R$ above the column $C_1$, finish if $k = 1$ or move the 1 from $c_{i+1}$ to the cell of $R$
above the column $C_2$ if $k > 1$.
\item Finally if there is no 1 in the cell of $R$ above the column $C_1$, move the 1 from $c_{i+1}$ to this cell.
\end{enumerate}

We continue by showing that the result of the transformation $f$ satisfies all five conditions of $\mathcal{G}_{i+1}(S)$.
\begin{enumerate}[(a)]
\item 
Since we only modified entries below $l(c_{i+1})$, the condition (a) is satisfied. 
\item The condition (b) could
only be broken by an $SE$-chain with its lower 1 inside $A$ or $C$, but the upper 1 of this chain would have 
formed an $SE$-chain with the 1 originally in the cell $c_{i+1}$, breaking the condition (a)
of $\mathcal{G}_i(S)$. Therefore, the condition (b) is satisfied. 
\item The condition (c) could only be broken
by an $SE$-chain with its upper 1 in $R$ or $c_{i+1}$, but the lower 1 of this chain together with the 1 originally in $c_{i+1}$
would break the condition (b) of $\mathcal{G}_i(S)$. 
\item The condition (d) could be broken in one of the following ways:
\begin{itemize} 
\item There is a $NE$-chain with the upper 1 in $R$ or $c_{i+1}$ and lower 1 in $A$. After
the performed transformation all 1's in $R$ are stricly left of or right above the column $C_1$, and a 1 remains in $c_{i+1}$
only if $A$ ends up empty, so this case cannot occur. 
\item There is a $NE$-chain with the upper 1 in $A$ and lower 1 outside $A$. In this case suppose that there is a $NE$-chain with
the upper 1 being the lowest 1 in the column $C_j$. The column $C_j$ was nonempty also before the transformation
and it was either the same or it contained the current filling of $C_{j+1}$, so the lowest 1 in $C_j$ was not higher than after the transformation,
therefore there was a $NE$-chain to begin with, which is a contradiction.
\item There is a $NE$-chain with the upper 1 in $C$. If the filling of $C$ has not changed, the $NE$-chain was there before the transformation also.
Otherwise $C$ now contains the filling of $C_k$. If the lower 1 of the $NE$-chain is strictly left of the column $C_k$, it forms
a $NE$-chain with a 1 in $C_k$ in the original filling. Otherwise it is below the rectangle $A$ and so it forms a $NE$-chain with the
1 in the cell $c_{i+1}$ in the original filling.
\end{itemize}

\item If the condition (e) is broken, the occurence of $S^{forb}(132)$ must have its upper $SE$-chain contained in $M$ and the lower 1
is outside of $M$ strictly southwest of it. Let $u$, $v$ be the two cells inside $M$ and let $w$ be the cell outside $M$ as illustrated
in Figure \ref{figure_sforb_occ}. Let $B$ be the maximal rectangle contained in $S$ with the cell $w$ as its southwest corner. S
Since there is a 1 in the cell $w$, all entries in the intersection of $M$ and $B$ are zero both before and after the transformation of the filling.
The cell $u$ is strictly above the rectangle $B$ and the cell $v$ is strictly right of the rectangle $B$. Now we observe an important property
of the described transformation: any row or column of $M$ was nonempty before the transformation if and only if it is nonempty after the transformation.
This is obvious for rows because we shift nonzero entries only in the horizontal direction. If the column $C_1$ is empty after the transformation,
then the cell of $R$ right above $C_1$ always contains a 1. If the column $C$ is nonempty after the transformation, either it was nonempty
before the transformation or there was a 1 in $c_{i+1}$ before the transformation. For other columns the discussion is straightforward.
We can use this property to deduce that there was a 1 in the same column of $M$ as $u$ and a 1 in the same row of $M$ as $v$ before the transformation.
Since these have to be outside the rectangle $B$, they form an occurence of $S^{forb}(132)$ together with the cell $w$ in the original filling
and a contradiction is reached. Therefore the condition (e) of $\mathcal{G}_{i+1}(S)$ could not have been broken by the transformation either.

\begin{figure}[h]
\centering
\begin{tikzpicture}[line cap=round,line join=round,>=triangle 45,x=0.8cm,y=0.8cm]
\clip(0.84,0.73) rectangle (14.25,12.24);
\draw (1,1)-- (5,1);
\draw (5,1)-- (5,2);
\draw (5,2)-- (8,2);
\draw (8,2)-- (8,4);
\draw (8,4)-- (11,4);
\draw (11,4)-- (11,6);
\draw (11,6)-- (13,6);
\draw (13,6)-- (13,9);
\draw (13,9)-- (14,9);
\draw (14,9)-- (14,12);
\draw (1,1)-- (1,4);
\draw (1,4)-- (2,4);
\draw (2,4)-- (2,8);
\draw (2,8)-- (4,8);
\draw (4,8)-- (4,11);
\draw (4,11)-- (8,11);
\draw (8,11)-- (8,12);
\draw (8,12)-- (14,12);
\draw [line width=1.6pt] (4,10)-- (10,10);
\draw [line width=1.6pt] (10,10)-- (10,4);
\draw [line width=1.6pt] (4,8)-- (4,4);
\draw [line width=1.6pt] (4,4)-- (8,4);
\draw [line width=1.6pt] (4,10)-- (4,8);
\draw [line width=1.6pt] (8,4)-- (10,4);
\draw [line width=1.6pt] (2,2)-- (8,2);
\draw [line width=1.6pt] (8,2)-- (8,8);
\draw [line width=1.6pt] (2,2)-- (2,8);
\draw (3,5) node {$B$};
\draw (6,6) node {\huge $\emptyset$};
\draw [line width=1.6pt] (2,8)-- (8,8);
\draw (6,8)--(6,9);
\draw (6,9)--(7,9);
\draw (7,9)--(7,8);
\draw (7,8)--(6,8);
\draw (6.5,8.5) node {$\times$};
\draw (6.76,8.76) node {$u$};
\draw (8,7)--(8,8);
\draw (8,8)--(9,8);
\draw (9,8)--(9,7);
\draw (9,7)--(8,7);
\draw (8.5,7.5) node {$\times$};
\draw (8.76,7.76) node{$v$};
\draw (2,2)--(2,3);
\draw (2,3)--(3,3);
\draw (3,3)--(3,2);
\draw (3,2)--(2,2);
\draw (2.5,2.5) node {$\times$};
\draw (2.76,2.76) node {$w$};
\draw (9,9)--(9,10);
\draw (9,10)--(10,10);
\draw (10,10)--(10,9);
\draw (10,9)--(9,9);
\draw (9.5,9.5) node {$c_{i+1}$};
\draw (9.11,4.71) node {$M$};
\end{tikzpicture}
\caption{An occurence of $S^{forb}(132)$ in the cells $u$, $v$ and $w$}
\label{figure_sforb_occ}
\end{figure}  
\end{enumerate}

We have shown that the transformation $f$ indeed transforms a filling of $\mathcal{G}^2_i(S)$ 
into a filling of $\mathcal{G}^2_{i+1}(S)$.
Next we describe the transformation $g$ 
transforms a filling of $\mathcal{G}^2_{i+1}(S)$ into a filling of $\mathcal{G}^2_i(S)$. Note that $g$
simply reverts the steps of the transformation $f$.

\begin{enumerate}
\item If there is exactly one entry 1 in $R$ and the column of $A$ below this entry is nonempty, move this entry to $c_{i+1}$ and finish.
\item Otherwise there are either at least two 1's in $R$ or the column below the single entry 1 is empty. In both cases we can
choose the rightmost empty column of $A$ such that there is a 1 in $R$ above it. We call this column $C_1$.
Label all nonempty columns of $A$ from left to right as $C_2, C_3, \ldots, C_k$. 
\item For each $1 \leq j < k$ copy the filling of $C_{j+1}$ to $C_j$, copy the filling of $C$ to $C_k$
and replace the filling of $C$ by all zeros.

\item If $k = 1$ finish. If $k > 2$ and there is a 1 in $R$ above $C_2$, this is the rightmost 1 in $R$, move it to $c_{i+1}$ and finish.

\item Finally if the entry 1 in $R$ above $C_1$ is the rightmost 1 in $R$, move this 1 to $c_{i+1}$.
\end{enumerate}

Next we choose any filling from $\mathcal{G}_{i+1}(S) \setminus \mathcal{G}_{i+1}^1(S)$, transform it using the described transformation $g$ and show
that it satisfies all five conditions for the fillings of $\mathcal{G}_i(S)$.

\begin{enumerate}[(a)]
\item The condition (a) could only be broken by a $SE$-chain with the lower 1 in the cell $c_{i+1}$, but then there is a $SE$-chain
present in the original filling with the same upper 1 and the lower 1 in the column $C$ violating
 the condition (b) of $\mathcal{G}_{i+1}(S)$, which is not possible.
\item All 1's that were moved in the transformation were moved to the left except the 1 in the cell $c_{i+1}$ which is above $l(c_i)$. Therefore
if the condition (b) is broken after the transformation, it must be broken in the original filling as well.
\item The condition (c) could only be broken by a $SE$-chain with the upper 1 in $R$ and the lower 1 in $C$, since otherwise it would
have been in the original filling. But after the transformation, either $R$ or $C$ is empty.
\item Suppose that after the transformation there is a $NE$-chain below $l(c_i)$. Then due to the properties of the transformation
one of them is inside of $M$, and the other is outside of $M$. Label the cell with the 1 inside $M$ as $v$
and the cell outside of $M$ as $u$. Again we will make use of the fact
that emptiness and nonemptiness of rows and columns of $M$ is preserved by the transformation.
If the column of $S$ in which lies $u$ intersects $M$, then the cell in the column of $M$ in which lies $v$ which was nonempty
in the original filling creates a $NE$-chain with $u$. Similarly if the row of $S$ in which lies $u$ intersects $M$, then the cell
in the row of $M$ in which lies $v$ which was nonempty in the original filling creates a $NE$-chain with $u$. Therefore
assume that neither the column or the row in which lies $u$ intersect $M$, thus $u$ lies strictly southwest of $M$.
Consider the maximal rectangle $B$ contained in $S$ with $u$ for its southwest corner. Clearly the intersection of $M$ and $B$
was empty before the transformation and it contains at least the 1 in the cell $u$ now. Let $z$ be the cell in the same column of $M$
as $v$ which was nonempty before the transformation and let $w$ be the cell in the same row of $M$ as $v$ which was nonempty before the transformation.
Then both $z$ and $w$ must lie outside of $B$ and therefore they form an occurence of $S^{forb}(132)$ together with $u$ in the original filling.

\begin{figure}[h]
\centering
\begin{tikzpicture}[line cap=round,line join=round,>=triangle 45,x=0.8cm,y=0.8cm]
\clip(0.84,0.73) rectangle (14.25,12.24);
\draw (1,1)-- (5,1);
\draw (5,1)-- (5,2);
\draw (5,2)-- (8,2);
\draw (8,2)-- (8,4);
\draw (8,4)-- (11,4);
\draw (11,4)-- (11,6);
\draw (11,6)-- (13,6);
\draw (13,6)-- (13,9);
\draw (13,9)-- (14,9);
\draw (14,9)-- (14,12);
\draw (1,1)-- (1,4);
\draw (1,4)-- (2,4);
\draw (2,4)-- (2,8);
\draw (2,8)-- (4,8);
\draw (4,8)-- (4,11);
\draw (4,11)-- (8,11);
\draw (8,11)-- (8,12);
\draw (8,12)-- (14,12);
\draw [line width=1.6pt] (4,10)-- (10,10);
\draw [line width=1.6pt] (10,10)-- (10,4);
\draw [line width=1.6pt] (4,8)-- (4,4);
\draw [line width=1.6pt] (4,4)-- (8,4);
\draw [line width=1.6pt] (4,10)-- (4,8);
\draw [line width=1.6pt] (8,4)-- (10,4);
\draw [line width=1.6pt] (2,2)-- (8,2);
\draw [line width=1.6pt] (8,2)-- (8,8);
\draw [line width=1.6pt] (2,2)-- (2,8);
\draw (3,5) node {$B$};
\draw (6,6) node {\huge $\emptyset$};
\draw [line width=1.6pt] (2,8)-- (8,8);
\draw (6,8)--(6,9);
\draw (6,9)--(7,9);
\draw (7,9)--(7,8);
\draw (7,8)--(6,8);
\draw (6.5,8.5) node {$\times$};
\draw (6.76,8.76) node {$z$};
\draw (6,7)--(6,8);
\draw (6,8)--(7,8);
\draw (7,8)--(7,7);
\draw (7,7)--(6,7);
\draw (6.5,7.5) node {$\times$};
\draw (6.76,7.76) node {$v$};
\draw (8,7)--(8,8);
\draw (8,8)--(9,8);
\draw (9,8)--(9,7);
\draw (9,7)--(8,7);
\draw (8.5,7.5) node {$\times$};
\draw (8.76,7.76) node{$w$};
\draw (2,2)--(2,3);
\draw (2,3)--(3,3);
\draw (3,3)--(3,2);
\draw (3,2)--(2,2);
\draw (2.5,2.5) node {$\times$};
\draw (2.76,2.76) node {$u$};
\draw (9,9)--(9,10);
\draw (9,10)--(10,10);
\draw (10,10)--(10,9);
\draw (10,9)--(9,9);
\draw (9.5,9.5) node {$c_{i+1}$};
\draw (9.11,4.71) node {$M$};
\end{tikzpicture}

\caption{An occurence of $S^{forb}(132)$ in the cells $u$, $w$ and $z$}
\end{figure} 

 
\item The condition (e) can be verified easily using the same approach as in the discussion of the map $f$.
\end{enumerate}

Overall we have shown that $f$ maps $\mathcal{G}^2_i(S)$ into $\mathcal{G}^2_{i+1}(S)$ and that $g$ maps $\mathcal{G}^2_{i+1}(S)$ into
$\mathcal{G}^2_i(S)$. In addition, since the transformations are carefully constructed so that one performs the exact opposite of the
other, we get that $fg = \text{id}$ and $gf = \text{id}$, which implies that $g = f^{-1}$ and $f$ is the bijection we were looking for.

\end{proof}
\begin{proof}[Proof of Theorem \ref{thm_2}]
Let $N$ be the number of cells of $S$.
From Lemma \ref{lemma_gi} we immediately get that there is a bijection between the fillings of $\mathcal{G}_1(S)$ and $\mathcal{G}_N(S)$.
As discussed above, $\mathcal{G}_1(S)$ is exactly the set of fillings avoiding a $SE$-chain of length 2 and $\mathcal{G}_N(S)$
is exactly the set of fillings avoiding $S^{forb}(132)$ and a $NE$-chain of length 2, which completes the proof. 
\end{proof}
