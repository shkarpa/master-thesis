\chapter*{Introduction}
\addcontentsline{toc}{chapter}{Introduction}


A permutation $\pi$ is said to be \emph{contained} in a permutation $\sigma$,
if some rows and columns of the permutation matrix of $\sigma$ can be removed to obtain the permutation matrix of $\pi$,
otherwise $\sigma$ \emph{avoids} $\pi$. A set of permutations closed under containment is called a \emph{permutation class},
and it is a \emph{singleton class} if, in addition, it can be described as the set of all permutations avoiding a single permutation $\pi$.
Such a singleton class is denoted by $\text{Av}(\pi)$.
The study and enumeration of pattern-avoiding permutations and permutation classes has now been an active field for several decades.
One of the important concepts studied in this field is \emph{Wilf-equivalence}. Two permutation classes are \emph{Wilf-equivalent},
if for every positive integer $n$ they contain the same number of permutations of length $n$.

The search for new Wilf-equivalent classes has led to the investigation of a stronger kind of equivalence. Consider a Ferrers diagram
whose every cell is filled with a 0 or a 1 such that there is at most one 1 in every row and every column. Such a filling
of the Ferrers diagram is called \emph{sparse} and it \emph{contains}
a permutation $\sigma$, if some rows and columns of the Ferrers diagram can be removed to obtain the permutation matrix of $\sigma$,
otherwise it \emph{avoids} $\sigma$.
We then say that two singleton classes $\text{Av}(\pi)$ and $\text{Av}(\sigma)$ are \emph{shape-Wilf-equivalent} if for every
Ferrers diagram the number of sparse fillings avoiding $\pi$ is the same as the number of sparse fillings avoiding $\sigma$. 
The most notable example of shape-Wilf-equivalence was found by Backelin, West and Xin \cite{Backelin07}, who have
shown that for any $k$, the classes $\text{Av}(12\cdots k)$ and $\text{Av}(k(k-1)\cdots21)$ are shape-Wilf-equvalent.
Later, Krattenthaler \cite{Krattenthaler06} found a nice and simpler bijective proof of this result, and his work,
in return, was further generalized by Rubey \cite{Rubey11} who extended the bijection to more general diagrams.

In the present thesis, we prove two new results about 0-1-fillings of \emph{skew diagrams}, which are diagrams
obtained as difference of two Ferrers diagrams one of which contains the other. The thesis consists of three chapters.
In the first chapter, we introduce all of the necessary terminology and notation. In the second chapter,
we define a subclass of skew diagrams, the \emph{simple skew diagrams}, and 
make use of the approaches of Krattenthaler and Rubey to construct a bijection between sparse fillings
of a given simple skew diagram avoiding $12\cdots k$ and sparse fillings avoiding $k\cdots21$. In the final
chapter we generalise a result of Jelínek \cite{Jelinek08} and show that for every skew diagram,
there at least as many general fillings avoiding $12$ as there are fillings avoiding $21$.

